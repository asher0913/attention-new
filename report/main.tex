\documentclass[12pt,a4paper]{article}
\usepackage[a4paper,margin=1in]{geometry}
\usepackage{amsmath,amssymb}
\usepackage{graphicx}
\usepackage[hidelinks]{hyperref}
\usepackage{cite}
\usepackage{titlesec}
\usepackage{pgfgantt}
\usepackage{enumitem}
\usepackage{float}
% Chapter-style headings in article
\titleformat{\section}{\large\bfseries}{\thesection}{1em}{}
\titleformat{\subsection}{\normalsize\bfseries}{\thesubsection}{1em}{}
\titleformat{\subsubsection}{\normalsize\bfseries}{\thesubsubsection}{1em}{}

\begin{document}
\thispagestyle{empty}

\begin{center}



% --- University Logo ---
\includegraphics[width=0.5\textwidth]{nottingham-logo.png}

\vspace{3cm}

% --- Title ---
\Large \textbf{Project Proposal:}\\[6pt]
\Large \textbf{Investigation of Defense Mechanisms against Model Inversion Attacks}

\vspace{2cm}
\normalsize {Submitted \textbf{October, 2025}, in partial fulfillment of \\ the conditions for the award of the degree \bf{BSc Computer Science}.}\\
% --- Student Info ---
\vspace{1.5cm}
\large
\textbf{Yixuan ZHANG}\\[6pt]
\vspace{0.5cm}
\textbf{20513731}\\[6pt]
\vspace{0.5cm}
\textbf{hnyyz39@nottingham.edu.cn}

\vspace{1cm}

% --- Supervisor ---
\textbf{Supervised by Dr. Jianfeng REN}\\[1cm]
\vspace{3cm}
% --- Programme of Study ---
\textit{BSc (Hons) Computer Science}\\[8pt]
School of Computer Science\\University of Nottingham Ningbo China

\vfill
\end{center}
\newpage
% ----------------------------
% Abstract
% ----------------------------
\begin{abstract}
Collaborative (edge–cloud) inference splits a network into an on-device encoder and a cloud decoder; the uploaded intermediate features are vulnerable to model inversion attacks (MIAs). Xia et al. \cite{cem} formalized a defense by maximizing a Gaussian-mixture lower bound of the conditional entropy $H(x \mid z)$, but fitting high-dimensional mixtures is computationally heavy and sensitive to non-Gaussian feature geometry. This project targets a learnable, distribution-agnostic surrogate of $H(x \mid z)$ that preserves task accuracy while increasing inversion error. To date: (1) the public CEM baseline has been reproduced under the default CIFAR-10 split (VGG11-BN-SGM, cutlayer=4, noise variance 0.025, $\lambda=16$); (2) a gated-attention conditional-entropy surrogate was designed, using class-wise gated pooling and variance-based log-entropy penalty, raising attack MSE from 0.0436 to 0.0473 and lowering SSIM from 0.432 to 0.411 with only a minor accuracy drop (85.18\% to 84.34\%); (3) an exploratory Slot + Gated Cross-Attention surrogate (slot aggregation plus Flamingo-style gated cross-attention) reached 85.04\% accuracy but degraded privacy (MSE 0.0393, SSIM 0.459). The current evidence favors gated attention as a stronger privacy–utility trade-off. The slot-based route likely needs architectural fusion (e.g., shortcut or parallel coupling with gated pooling) to stabilize variance estimates before it can surpass the baseline. Next steps focus on such fusion designs and hyperparameter sweeps to retain accuracy while further elevating inversion error.
\end{abstract}
\newpage
\tableofcontents
\newpage

% =========================================================
\section{Introduction}

\subsection{Background and Motivation}
In collaborative (edge–cloud) inference, a lightweight encoder on the device produces an intermediate representation $z$ that is sent to a cloud-side decoder. This reduces on-device compute and bandwidth, but exposes $z$ to model inversion attacks (MIAs): adversaries with access to $z$ and the encoder can train decoders or generative models to reconstruct the private input $x$ \cite{fredrikson_mi,hitaj_gan_mi,nasr_whitebox,zhu_dlg,carlini_diffusion}. Prior obfuscation methods (noise injection, pruning, dropout, adversarial representation learning) improve privacy empirically \cite{noise_arl,patrol,distcorr,srivastava_dropout} but lack a principled link to worst-case inversion error.

Xia et al. \cite{cem} introduced Conditional Entropy Maximization (CEM), showing that higher $H(x \mid z)$ lower-bounds the minimal reconstruction MSE. They operationalize $H(x \mid z)$ with a Gaussian Mixture Model (GMM) surrogate, but high-dimensional, non-Gaussian features make GMM fitting unstable, costly, and sensitive to initialization. This motivates a learnable, distribution-agnostic surrogate that can be optimized end-to-end with the split model.

\subsection{Problem Statement}
We seek a surrogate of conditional entropy that is (i) differentiable and stable on high-dimensional, non-Gaussian features; (ii) compatible with split inference constraints (encoder light, decoder heavy); and (iii) empirically improves the privacy–utility trade-off against strong MIA decoders on CIFAR-10 under a standard split protocol.

\subsection{Aim and Objectives}
\textbf{Aim.} Develop and evaluate a learnable surrogate for conditional entropy that improves MIA robustness in split inference without materially degrading task accuracy.

\textbf{Objectives.}
\begin{itemize}[leftmargin=1.5em]
    \item Reproduce the CEM baseline \cite{cem} on CIFAR-10 with the public protocol (VGG11-BN-SGM, cutlayer=4, Gaussian noise).
    \item Design a gated-attention surrogate that replaces GMM fitting with class-wise gated pooling and variance-based entropy penalties.
    \item Explore a Slot + Gated Cross-Attention surrogate (slot aggregation plus gated cross-attention) as a richer, learnable mixture analogue.
    \item Benchmark the two surrogates against the baseline using classification accuracy and inversion metrics (MSE, SSIM, PSNR) under a common threat model.
    \item Analyze failure modes of the slot-based surrogate and propose fusion strategies (e.g., shortcut/parallel coupling) to stabilize variance estimation.
\end{itemize}

% =========================================================
\section{Related Work}
% 目标:展示你对领域“够懂”,并为你方法选择做铺垫。
%
% 建议按“主题”而不是“论文流水账”写:
% \subsection{Model Inversion Attacks}
% \subsection{Split Learning / Edge-Cloud Privacy}
% \subsection{Conditional Entropy Minimization and Surrogates}
% - 引出 baseline 用 GMM 的思路及其潜在问题(不可微/敏感/训练不稳等)。
% \subsection{Gated Attention Mechanisms}
% - 你的替代思路理论依据。
% \subsection{Slot Attention and Cross-Attention}
% - 说明你为什么认为它可能适合做更强的表征/聚类替代。

\subsection{Model Inversion Attacks}
% TODO

\subsection{Split Learning / Edge-Cloud Privacy}
% TODO

\subsection{Conditional Entropy Minimization and Surrogates}
% TODO

\subsection{Gated Attention Mechanisms}
% TODO

\subsection{Slot Attention and Cross-Attention}
% TODO

% =========================================================
\section{Methodology}
% 这是 20%+20% 的核心拿分区:不仅讲“是什么”,还要讲“为什么这样选”。
%
% 强烈建议放 1 张总览图:
% - baseline pipeline
% - 你的 gated-attention 替换点
% - 你的 slot+gated cross attention 分支(作为探索性路线)
%
% 建议结构:
% \subsection{Baseline Overview (CVPR Spotlight)}
% - 简述 baseline 架构与 GMM-based CEM loss 计算流程。
%
% \subsection{Proposed Method 1: Gated-Attention CEM}
% - 讲清 attention pooling、类内加权均值/方差、log-variance 约束。
% - 给出关键公式/伪代码。
% - 说明优点:可微、稳定、无需聚类初始化。
%
% \subsection{Proposed Method 2: Slot + Gated Cross-Attention CEM (Exploratory)}
% - 讲动机:期望更强结构化表征。
% - 讲你目前实现的配置/假设。
%
% \subsection{Evaluation Protocol}
% - 数据集、指标、默认训练脚本设置(此处只写原则,不必贴 shell)。
% - 说明公平对比原则。

\subsection{Baseline Overview (CVPR Spotlight)}
% TODO

\subsection{Proposed Method 1: Gated-Attention CEM}
% TODO
% You may add a short equation block here later.

\subsection{Proposed Method 2: Slot + Gated Cross-Attention CEM (Exploratory)}
% TODO

\subsection{Evaluation Protocol}
% TODO

% =========================================================
\section{Implementation}
% 说明“你真的做了”。
% 可写:
% - 代码结构(模块级别)
% - 关键工程决策
% - 训练稳定性处理(warmup, loss scaling, gradient handling 等)
%
% 这部分可以简单,但要让评阅人看到可复现性意识。

% TODO

% =========================================================
\section{Preliminary Results and Analysis}
% 中期报告不求结果很大,但要“有证据、有对照、有解释”。
%
% 强烈建议一个主表:
% Table: CIFAR-10 (default protocol)
%   - Baseline (GMM surrogate)
%   - Yours (Gated-Attention CEM)
%   - Slot+Gated Cross-Attention (default setting)
%
% 再加 1 小节解释:
% - 为什么 1 更好(从优化稳定性/可微性/类内方差控制角度)
% - 为什么 2 目前不如 baseline(可能原因列表)
%   例:slot 数/初始化、cross-attn 门控位置、loss 权重耦合、训练资源不足等
% - 清晰声明:目前只在默认脚本上测试的范围边界。

\subsection{Main Quantitative Comparison}
% TODO: add a table later.

\subsection{Why Method 1 Improves Over the Baseline}
% TODO

\subsection{Why Method 2 Underperforms So Far}
% TODO: list hypotheses + supporting observations.

% =========================================================
\section{Progress Against Workplan}
% 必写:原计划 vs 实际。
% 你可以用:
% - 一个小表格(Task / Planned / Achieved / Notes)
% - 或 Gantt 图(更直观)
%
% 这里要写“可量化进展”。

\subsection{Completed Work}
\begin{itemize}[leftmargin=1.5em]
    \item % TODO: Reproduced/understood baseline pipeline.
    \item % TODO: Implemented gated-attention CEM surrogate.
    \item % TODO: Ran CIFAR-10 default protocol and observed improvement.
    \item % TODO: Implemented Slot + Gated Cross-Attention version (initial attempt).
\end{itemize}

\subsection{Original Plan vs Current Status}
% TODO: add a small table.

\subsection{Updated Plan}
% TODO: bullet milestones for Jan--Apr.
% Example milestones:
% M1: Hyper-parameter sweep for Method 1 and 2.
% M2: Fusion design (parallel/serial/shortcut).
% M3: Expanded evaluation (more datasets/attacks).
% M4: Ablations + write-up.

% Optional Gantt placeholder
% \begin{figure}[h]
% \centering
% \includegraphics[width=0.9\textwidth]{gantt_placeholder.png}
% \caption{Updated Gantt chart for the remainder of the project.}
% \end{figure}

% =========================================================
\section{Reflection and Risk Management}
% 这是把“做项目的人味”写出来的地方,也是 rubric 高分点。
%
% 必写:
% - 你遇到的真实问题
% - 你怎么解决
% - 你学到了什么
% - 风险与缓解策略(最好 3--5 条)

\subsection{Key Challenges So Far}
\begin{itemize}[leftmargin=1.5em]
    \item % TODO
    \item % TODO
\end{itemize}

\subsection{What I Learned}
% TODO

\subsection{Risks and Mitigation}
\begin{itemize}[leftmargin=1.5em]
    \item \textbf{Risk:} % TODO
          \textbf{Mitigation:} % TODO
    \item \textbf{Risk:} % TODO
          \textbf{Mitigation:} % TODO
\end{itemize}

% =========================================================
\section{Conclusion}
% 3--6 句即可:
% - 重申 aim
% - 总结目前最硬的结论:Method 1 works better under default CIFAR-10.
% - Method 2 is promising but currently underperforms under default setting.
% - 下一步明确聚焦“融合两者 + 更系统评测”。

% TODO

% =========================================================
\bibliographystyle{plain}
\bibliography{references}

% If you don't want BibTeX yet, you can comment above and use manual references.

% =========================================================
\appendix
\section{Additional Details}
% 长代码/更多结果/更多图放这里。

\end{document}
