\documentclass[12pt,a4paper]{article}
\usepackage[a4paper,margin=1in]{geometry}
\usepackage{amsmath,amssymb}
\usepackage{graphicx}
\usepackage[hidelinks]{hyperref}
\usepackage{cite}
\usepackage{titlesec}
\usepackage{pgfgantt}
\usepackage{enumitem}
\usepackage{float}
% Chapter-style headings in article
\titleformat{\section}{\large\bfseries}{\thesection}{1em}{}
\titleformat{\subsection}{\normalsize\bfseries}{\thesubsection}{1em}{}
\titleformat{\subsubsection}{\normalsize\bfseries}{\thesubsubsection}{1em}{}

\begin{document}
\thispagestyle{empty}

\begin{center}



% --- University Logo ---
\includegraphics[width=0.5\textwidth]{nottingham-logo.png}

\vspace{3cm}

% --- Title ---
\Large \textbf{Project Proposal:}\\[6pt]
\Large \textbf{Investigation of Defense Mechanisms against Model Inversion Attacks}

\vspace{2cm}
\normalsize {Submitted \textbf{October, 2025}, in partial fulfillment of \\ the conditions for the award of the degree \bf{BSc Computer Science}.}\\
% --- Student Info ---
\vspace{1.5cm}
\large
\textbf{Yixuan ZHANG}\\[6pt]
\vspace{0.5cm}
\textbf{20513731}\\[6pt]
\vspace{0.5cm}
\textbf{hnyyz39@nottingham.edu.cn}

\vspace{1cm}

% --- Supervisor ---
\textbf{Supervised by Dr. Jianfeng REN}\\[1cm]
\vspace{3cm}
% --- Programme of Study ---
\textit{BSc (Hons) Computer Science}\\[8pt]
School of Computer Science\\University of Nottingham Ningbo China

\vfill
\end{center}
\newpage
% ----------------------------
% Abstract
% ----------------------------
\begin{abstract}
% 150--250 words.
% 1) 1--2 sentences: background + why MIA matters in split/edge-cloud setting.
% 2) 1 sentence: your aim.
% 3) 2--3 sentences: what you have done so far:
%    - Main: replaced GMM-based CEM surrogate with gated-attention CEM.
%    - Result: improved performance under CIFAR-10 default settings.
%    - Secondary: explored Slot + Gated Cross-Attention CEM variant.
% 4) 1 sentence: key finding about (2) not outperforming baseline so far.
% 5) 1 sentence: next steps: how you plan to fuse/position both ideas.
\end{abstract}
\newpage
\tableofcontents
\newpage

% =========================================================
\section{Introduction}
% 目的:让评阅人快速知道你在解决什么、为什么值得做、你要怎么做。
%
% 建议小结构:
% \subsection{Background and Motivation}
% - Model inversion attack 简述(1段)。
% - Edge-cloud / split learning 场景为什么更敏感(1段)。
%
% \subsection{Problem Statement}
% - 明确你要防的是什么攻击、在哪个设置下评测。
%
% \subsection{Aim and Objectives}
% - Aim 用 1 句清晰话。
% - Objectives 用 4--6 条 bullet(可检验)。
%   例:
%   O1: Re-implement/understand CVPR Spotlight baseline CEM pipeline.
%   O2: Design a gated-attention surrogate to replace GMM for conditional entropy loss.
%   O3: Evaluate on CIFAR-10 with default protocol and compare to baseline.
%   O4: Explore Slot + Gated Cross-Attention CEM extension.
%   O5: Analyze failure modes and propose fusion strategies for next stage.

\subsection{Background and Motivation}
% TODO

\subsection{Problem Statement}
% TODO

\subsection{Aim and Objectives}
\textbf{Aim.} % TODO one sentence.

\textbf{Objectives.}
\begin{itemize}[leftmargin=1.5em]
    \item % TODO O1    \item % TODO O2
    \item % TODO O3
    \item % TODO O4
    \item % TODO O5
\end{itemize}

% =========================================================
\section{Related Work}
% 目标:展示你对领域“够懂”,并为你方法选择做铺垫。
%
% 建议按“主题”而不是“论文流水账”写:
% \subsection{Model Inversion Attacks}
% \subsection{Split Learning / Edge-Cloud Privacy}
% \subsection{Conditional Entropy Minimization and Surrogates}
% - 引出 baseline 用 GMM 的思路及其潜在问题(不可微/敏感/训练不稳等)。
% \subsection{Gated Attention Mechanisms}
% - 你的替代思路理论依据。
% \subsection{Slot Attention and Cross-Attention}
% - 说明你为什么认为它可能适合做更强的表征/聚类替代。

\subsection{Model Inversion Attacks}
% TODO

\subsection{Split Learning / Edge-Cloud Privacy}
% TODO

\subsection{Conditional Entropy Minimization and Surrogates}
% TODO

\subsection{Gated Attention Mechanisms}
% TODO

\subsection{Slot Attention and Cross-Attention}
% TODO

% =========================================================
\section{Methodology}
% 这是 20%+20% 的核心拿分区:不仅讲“是什么”,还要讲“为什么这样选”。
%
% 强烈建议放 1 张总览图:
% - baseline pipeline
% - 你的 gated-attention 替换点
% - 你的 slot+gated cross attention 分支(作为探索性路线)
%
% 建议结构:
% \subsection{Baseline Overview (CVPR Spotlight)}
% - 简述 baseline 架构与 GMM-based CEM loss 计算流程。
%
% \subsection{Proposed Method 1: Gated-Attention CEM}
% - 讲清 attention pooling、类内加权均值/方差、log-variance 约束。
% - 给出关键公式/伪代码。
% - 说明优点:可微、稳定、无需聚类初始化。
%
% \subsection{Proposed Method 2: Slot + Gated Cross-Attention CEM (Exploratory)}
% - 讲动机:期望更强结构化表征。
% - 讲你目前实现的配置/假设。
%
% \subsection{Evaluation Protocol}
% - 数据集、指标、默认训练脚本设置(此处只写原则,不必贴 shell)。
% - 说明公平对比原则。

\subsection{Baseline Overview (CVPR Spotlight)}
% TODO

\subsection{Proposed Method 1: Gated-Attention CEM}
% TODO
% You may add a short equation block here later.

\subsection{Proposed Method 2: Slot + Gated Cross-Attention CEM (Exploratory)}
% TODO

\subsection{Evaluation Protocol}
% TODO

% =========================================================
\section{Implementation}
% 说明“你真的做了”。
% 可写:
% - 代码结构(模块级别)
% - 关键工程决策
% - 训练稳定性处理(warmup, loss scaling, gradient handling 等)
%
% 这部分可以简单,但要让评阅人看到可复现性意识。

% TODO

% =========================================================
\section{Preliminary Results and Analysis}
% 中期报告不求结果很大,但要“有证据、有对照、有解释”。
%
% 强烈建议一个主表:
% Table: CIFAR-10 (default protocol)
%   - Baseline (GMM surrogate)
%   - Yours (Gated-Attention CEM)
%   - Slot+Gated Cross-Attention (default setting)
%
% 再加 1 小节解释:
% - 为什么 1 更好(从优化稳定性/可微性/类内方差控制角度)
% - 为什么 2 目前不如 baseline(可能原因列表)
%   例:slot 数/初始化、cross-attn 门控位置、loss 权重耦合、训练资源不足等
% - 清晰声明:目前只在默认脚本上测试的范围边界。

\subsection{Main Quantitative Comparison}
% TODO: add a table later.

\subsection{Why Method 1 Improves Over the Baseline}
% TODO

\subsection{Why Method 2 Underperforms So Far}
% TODO: list hypotheses + supporting observations.

% =========================================================
\section{Progress Against Workplan}
% 必写:原计划 vs 实际。
% 你可以用:
% - 一个小表格(Task / Planned / Achieved / Notes)
% - 或 Gantt 图(更直观)
%
% 这里要写“可量化进展”。

\subsection{Completed Work}
\begin{itemize}[leftmargin=1.5em]
    \item % TODO: Reproduced/understood baseline pipeline.
    \item % TODO: Implemented gated-attention CEM surrogate.
    \item % TODO: Ran CIFAR-10 default protocol and observed improvement.
    \item % TODO: Implemented Slot + Gated Cross-Attention version (initial attempt).
\end{itemize}

\subsection{Original Plan vs Current Status}
% TODO: add a small table.

\subsection{Updated Plan}
% TODO: bullet milestones for Jan--Apr.
% Example milestones:
% M1: Hyper-parameter sweep for Method 1 and 2.
% M2: Fusion design (parallel/serial/shortcut).
% M3: Expanded evaluation (more datasets/attacks).
% M4: Ablations + write-up.

% Optional Gantt placeholder
% \begin{figure}[h]
% \centering
% \includegraphics[width=0.9\textwidth]{gantt_placeholder.png}
% \caption{Updated Gantt chart for the remainder of the project.}
% \end{figure}

% =========================================================
\section{Reflection and Risk Management}
% 这是把“做项目的人味”写出来的地方,也是 rubric 高分点。
%
% 必写:
% - 你遇到的真实问题
% - 你怎么解决
% - 你学到了什么
% - 风险与缓解策略(最好 3--5 条)

\subsection{Key Challenges So Far}
\begin{itemize}[leftmargin=1.5em]
    \item % TODO
    \item % TODO
\end{itemize}

\subsection{What I Learned}
% TODO

\subsection{Risks and Mitigation}
\begin{itemize}[leftmargin=1.5em]
    \item \textbf{Risk:} % TODO
          \textbf{Mitigation:} % TODO
    \item \textbf{Risk:} % TODO
          \textbf{Mitigation:} % TODO
\end{itemize}

% =========================================================
\section{Conclusion}
% 3--6 句即可:
% - 重申 aim
% - 总结目前最硬的结论:Method 1 works better under default CIFAR-10.
% - Method 2 is promising but currently underperforms under default setting.
% - 下一步明确聚焦“融合两者 + 更系统评测”。

% TODO

% =========================================================
\bibliographystyle{plain}
\bibliography{references}

% If you don't want BibTeX yet, you can comment above and use manual references.

% =========================================================
\appendix
\section{Additional Details}
% 长代码/更多结果/更多图放这里。

\end{document}