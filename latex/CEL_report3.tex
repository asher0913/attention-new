\documentclass[12pt]{article}

\usepackage[UTF8]{ctex}
\usepackage{geometry}
\usepackage{amsmath, amssymb, amsfonts, bm}
\usepackage{mathtools}
\usepackage{physics}
\usepackage{graphicx}
\usepackage{tikz}
\usepackage{subcaption}
\usetikzlibrary{positioning}
\usepackage{pgfplots}
\usepackage{enumitem}
\usepackage{hyperref}
\usepackage{array}
\usepackage{booktabs}

\geometry{a4paper, margin=2.2cm}
\setlength{\parskip}{0.9em}
\setlength{\parindent}{2em}
\pgfplotsset{compat=1.18}

\title{门控注意力条件熵正则的全流程解析 \\[3pt]
\large ——从原始数据到条件熵惩罚的逐步讲解}
\author{}
\date{\today}

\begin{document}

\maketitle
\tableofcontents
\newpage

\section{导言:我们要解决什么问题?}

在协同推理(split inference)场景中,客户端本地运行前端神经网络 $f_{\bm{\theta}}$,将输入图像 $x$ 映射成中间表示 $z = f_{\bm{\theta}}(x)$;该表示(也称 smashed data)被发送到服务器继续推理。若攻击者截获 $z$,有可能通过模型反演(Model Inversion)或属性推断等方法恢复原始图像或敏感特征。为了降低风险,我们希望让相同类别的 smashed data ``长得尽量像'',也就是让条件熵 $H(Z \mid Y)$ 尽可能小。这样一来,攻击者难以从 $z$ 推断出具体的原始图像细节。

本报告聚焦于 \texttt{gated-att} 实现中门控注意力条件熵正则(Conditional Entropy Loss, CEL)的计算过程:从原始 mini-batch 数据开始,一直到得到具体的正则值 $\mathcal{L}_{\text{CEL}}$,并详细解释注意力与门控在其中扮演的角色。本文不再讨论其他衍生内容(例如训练建议、复现流程等),目的是确保读者掌握这一套机制的来龙去脉。

\section{基本符号与数据结构}

\begin{itemize}[leftmargin=2.3em]
    \item $x \in \mathcal{X}$:客户端输入(如图像);$y \in \{1,\dots,C\}$:对应标签。
    \item $f_{\bm{\theta}}$:客户端编码器,输出 smashed data $z = f_{\bm{\theta}}(x) \in \mathbb{R}^d$。
    \item mini-batch $\mathcal{B} = \{ (x_i, y_i) \}_{i=1}^{B}$,其 smashed data 集合 $Z_{\mathcal{B}} = \{ z_i \}_{i=1}^{B}$。
    \item 对于标签 $c$,记 $Z_{\mathcal{B}}^{(c)} = \{ z_i \mid y_i = c \}$,其大小为 $m_c$。
    \item 总损失 $\mathcal{L} = \mathcal{L}_{\text{CE}} + \lambda \mathcal{L}_{\text{CEL}}$,其中 $\mathcal{L}_{\text{CE}}$ 是交叉熵,$\mathcal{L}_{\text{CEL}}$ 是本文关注的正则项。
    \item $\tau > 0$:方差阈值,$\varepsilon > 0$:防止取对数时为零的平滑项。
    \item 注意力隐层维度 $h$;权重矩阵 $W_V, W_U \in \mathbb{R}^{h \times d}$;向量 $\bm{w} \in \mathbb{R}^h$。
\end{itemize}

\section{整体流程概览}

对当前 mini-batch $\mathcal{B}$,门控注意力 CEL 的计算分为以下步骤:

\begin{enumerate}[label=\textbf{步骤 \arabic*},leftmargin=2.5em]
    \item 客户端编码器生成 smashed data:$z_i = f_{\bm{\theta}}(x_i)$。
    \item (可选)对 $z_i$ 进行规范化(如 LayerNorm),得到 $\tilde{z}_i$。
    \item 通过门控注意力网络,为同一类别样本生成 softmax 注意力权重 $\alpha_i$。
    \item 根据 $\alpha_i$ 计算加权均值 $\bar{z}_c$ 和加权方差 $v_c$。
    \item 将 $v_c$ 代入门控函数,得到类别 $c$ 的惩罚 $\mathcal{R}(c)$。
    \item 聚合得出 $\mathcal{L}_{\text{CEL}} = \sum_{c} \beta_c \mathcal{R}(c)$。
\end{enumerate}

核心困难在于第 3 步与第 4 步:注意力如何生成,门控如何调节,以及它们如何共同约束 smashed data 的散布。下面详细展开。

\section{注意力与门控的作用机理}

\subsection{为什么需要注意力?}

若我们对同一类别的 smashed data 简单地计算无权方差
\[
    v_c^{\text{plain}} = \frac{1}{m_c} \sum_{i=1}^{m_c} \left\| z_i - \frac{1}{m_c} \sum_{j=1}^{m_c} z_j \right\|_2^2,
\]
这隐式假设所有样本同等可靠。但在实践中:
\begin{itemize}[leftmargin=2.3em]
    \item mini-batch 中可能有噪声或异常值;
    \item 同类样本可能分布在多个子簇,单一均值无法概括全部模式;
    \item 我们希望自动识别“更典型”的样本,并让它们主导方差估计。
\end{itemize}
注意力机制正好提供一种可微的、数据驱动的权重分配方式:网络会学习谁更重要。

\subsection{注意力模块结构}

对任意 smashed data $z \in \mathbb{R}^d$,注意力网络(图 \ref{fig:att_pipeline})执行以下操作:

\begin{figure}[htbp]
    \centering
    \begin{tikzpicture}[>=stealth, node distance=2.5cm]
        \node (z) at (0,0) {$z$};
        \node (ln) at (2.2,0) {LayerNorm};
        \node (V) at (5.2,1.2) {$\tanh(W_V \tilde{z})$};
        \node (U) at (5.2,-1.2) {$\sigma(W_U \tilde{z})$};
        \node (mul) at (8.4,0) {$\bm{s} = \bm{v} \odot \bm{u}$};
        \node (w) at (11.2,0) {$\alpha' = \bm{w}^\top \bm{s}$};
        \node (sm) at (14.0,0) {softmax $\alpha$};

        \draw[->] (z) -- (ln);
        \draw[->] (ln) -- node[above] {$\tilde{z}$} (V);
        \draw[->] (ln) -- (U);
        \draw[->] (V) -- (mul);
        \draw[->] (U) -- (mul);
        \draw[->] (mul) -- (w);
        \draw[->] (w) -- (sm);
    \end{tikzpicture}
    \caption{门控注意力前向流程:双投影 + 元素乘法 + 线性汇聚 + softmax}
    \label{fig:att_pipeline}
\end{figure}

\paragraph{第 1 步:规范化。}
为减少不同样本尺度差异的影响,通常先对 $z$ 做 LayerNorm:
\[
    \tilde{z} = \frac{z - \mu_z}{\sigma_z}, \quad \text{其中} \quad \mu_z = \frac{1}{d}\sum_{k=1}^{d} z_k,\;
    \sigma_z = \sqrt{\frac{1}{d}\sum_{k=1}^{d} (z_k - \mu_z)^2 + \epsilon_{\text{LN}}}.
\]

\paragraph{第 2 步:双投影。}
\[
    \bm{v} = \tanh(W_V \tilde{z}), \qquad \bm{u} = \sigma(W_U \tilde{z}).
\]
其中 $W_V$ 学习如何在 $\tanh$ 的范围内提取关键信息;$W_U$ 经由 Sigmoid 输出各维度的门控值(0 到 1)。

\paragraph{第 3 步:门控融合。}
\[
    \bm{s} = \bm{v} \odot \bm{u},
\]
逐元素乘法意味着:若某一维度在 $\bm{u}$ 中值较低,该维度的 $\bm{v}$ 被抑制;反之则保留。这样可以动态过滤掉不可靠或噪声特征。

\paragraph{第 4 步:线性汇聚与 softmax。}
\[
    \alpha' = \bm{w}^\top \bm{s}, \qquad
    \alpha = \frac{\exp(\alpha')}{\sum_{j} \exp(\alpha'_j)}.
\]
紧接着对所有同类样本的 $\alpha'$ 做 softmax,得到归一化注意力权重 $\alpha_i$。此权重满足 $\sum_i \alpha_i = 1$,且可微。

\subsection{为什么门控有效?}

\begin{itemize}[leftmargin=2.3em]
    \item \textbf{可解释性}:$W_U$ 生成的 Sigmoid 输出可以理解为“是否让该特征通过”;如果模型判定某样本上的特定特征不可靠,就会通过门控削弱该特征在 $\bm{v}$ 中的贡献。
    \item \textbf{稳定性}:softmax 确保注意力权重总和为 1,避免了异常权重出现梯度爆炸。
    \item \textbf{可学习性}:$W_V$, $W_U$, $\bm{w}$ 均由数据驱动学习,使得注意力聚焦在能减小方差、保持分类性能的样本上。
\end{itemize}

\section{从注意力到条件熵惩罚:数学推导}

\subsection{加权均值与加权方差}

得到权重 $\{\alpha_i\}_{i=1}^{m_c}$ 后,定义加权均值
\[
    \bar{z}_c = \sum_{i=1}^{m_c} \alpha_i z_i.
\]
加权方差(即类内散度)则为
\begin{equation}
    v_c = \sum_{i=1}^{m_c} \alpha_i \norm{z_i - \bar{z}_c}_2^2.
    \label{eq:variance}
\end{equation}
如果将 $\alpha_i$ 理解为样本重要性,$v_c$ 就是重要性加权的均方偏差。

\subsection{门控函数与阈值}

为了避免 $v_c$ 过小或过大带来数值问题,引入阈值门控:
\[
    \mathcal{R}(c) = \max\bigl(0, \log(v_c + \varepsilon) - \log(\tau + \varepsilon)\bigr).
\]
\begin{itemize}[leftmargin=2.3em]
    \item 若 $v_c \le \tau$,说明类内散度已经足够小,惩罚为 0;
    \item 若 $v_c > \tau$,惩罚随 $\log(v_c)$ 增大,鼓励网络进一步压缩 smashed data;
    \item $\varepsilon$ 防止取对数时出现 $\log 0$。
\end{itemize}

最终,对整个 batch 的 CEL 为
\begin{equation}
    \mathcal{L}_{\text{CEL}} = \sum_{c=1}^{C} \beta_c \, \mathcal{R}(c),
    \qquad \beta_c = \frac{m_c}{B},
    \label{eq:cel}
\end{equation}
即以类别在 batch 中的占比作为权重。

\subsection{梯度如何影响模型?}

当某类别方差 $v_c$ 超过阈值时,$\mathcal{R}(c)$ 的梯度开始作用于:
\begin{itemize}[leftmargin=2.3em]
    \item 编码器参数 $\bm{\theta}$:推动 smashed data 收缩;
    \item 注意力参数 $W_V, W_U, \bm{w}$:调整注意力权重,让有代表性的样本更受重视。
\end{itemize}

举例说明,如果第 $i$ 个样本与均值相差较大,则 $\norm{z_i - \bar{z}_c}_2^2$ 较大,梯度会尝试:
\begin{itemize}
    \item 减小此偏差(即推动编码器让 $z_i$ 更接近 $\bar{z}_c$);
    \item 或降低 $\alpha_i$(如果模型认为该样本是一枚“坏样本”)。
\end{itemize}
这正体现了注意力 + 门控的配合:网络可自主决定是拉近样本还是降低其重要性。

\section{完整计算示例}

为了让本科生也能跟上,我们构建一个小型示例,展示每个数值是如何算出来的。

\subsection{数据设定}

考虑一个 mini-batch,有两个类别,每类三个 smashed data(二维向量):
\[
\begin{aligned}
    Z^{(1)} &= \{ (0.0, 0.0),\; (0.2, 0.1),\; (-0.1, 0.05) \}, \\
    Z^{(2)} &= \{ (1.0, 1.0),\; (0.9, 1.1),\; (1.2, 0.8) \}.
\end{aligned}
\]
取注意力隐层维度 $h=2$,设置
\[
    W_V = \begin{bmatrix} 1 & 0 \\ 0 & 1 \end{bmatrix}, \quad
    W_U = \begin{bmatrix} 0.5 & 0 \\ 0 & 0.5 \end{bmatrix}, \quad
    \bm{w} = \begin{bmatrix} 1 \\ 1 \end{bmatrix}.
\]
阈值 $\tau = 0.02$,平滑项 $\varepsilon = 10^{-6}$。

\subsection{类别 1 的注意力计算}

\paragraph{1. 规范化。} 为简化说明,我们假设 LayerNorm 近似输出原值(即 $\tilde{z} \approx z$)。

\paragraph{2. 投影与门控。}
\[
\begin{aligned}
    \bm{v}_1 &= \tanh((0.0, 0.0)) = (0, 0), \\
    \bm{u}_1 &= \sigma(0.5 \cdot (0.0, 0.0)) = (0.5, 0.5), \\
    \bm{s}_1 &= (0,0) \odot (0.5,0.5) = (0,0), \\
    \alpha'_1 &= \bm{w}^\top \bm{s}_1 = 0.
\end{aligned}
\]
同理,
\[
\begin{aligned}
    \bm{v}_2 &= \tanh((0.2, 0.1)) \approx (0.197, 0.0997), \\
    \bm{u}_2 &= \sigma(0.5 \cdot (0.2, 0.1)) \approx (0.55, 0.525), \\
    \bm{s}_2 &\approx (0.108, 0.052), \\
    \alpha'_2 &= 0.160.
\end{aligned}
\]
\[
\begin{aligned}
    \bm{v}_3 &= \tanh((-0.1, 0.05)) \approx (-0.0997, 0.04996), \\
    \bm{u}_3 &= \sigma(0.5 \cdot (-0.1, 0.05)) \approx (0.475, 0.5125), \\
    \bm{s}_3 &\approx (-0.0473, 0.0256), \\
    \alpha'_3 &= -0.0217.
\end{aligned}
\]

\paragraph{3. softmax 权重。}
\[
\begin{aligned}
    \alpha_1 &= \frac{e^{0}}{e^{0} + e^{0.160} + e^{-0.0217}} \approx 0.322, \\
    \alpha_2 &\approx 0.381, \\
    \alpha_3 &\approx 0.297.
\end{aligned}
\]

\paragraph{4. 加权均值与方差。}
\[
\bar{z}_1 = 0.322(0,0) + 0.381(0.2,0.1) + 0.297(-0.1, 0.05) \approx (0.046, 0.053).
\]
\[
\begin{aligned}
    v_1 &= 0.322 \norm{(0,0) - (0.046,0.053)}^2 + 0.381 \norm{(0.2,0.1) - (0.046,0.053)}^2 \\
        &\quad + 0.297 \norm{(-0.1,0.05) - (0.046,0.053)}^2 \\
        &\approx 0.322 (0.0049 + 0.0028) + 0.381 (0.0237 + 0.0022) + 0.297(0.0214 + 0.00001) \\
        &\approx 0.0024 + 0.010 + 0.0064 \approx 0.0188.
\end{aligned}
\]

\paragraph{5. 门控惩罚。}
由于 $v_1 = 0.0188 < \tau = 0.02$,因此 $\mathcal{R}(1) = 0$,说明类别 1 的散度已经足够小。

\subsection{类别 2 的计算(概述)}

同理可得类别 2 的权重 $\alpha_i$ 大约分布在 $(0.34, 0.33, 0.33)$ 左右,加权均值 $\bar{z}_2 \approx (1.03, 0.96)$,加权方差 $v_2 \approx 0.0215$。因为 $v_2 > \tau$,有
\[
\mathcal{R}(2) = \log(0.0215 + 10^{-6}) - \log(0.02 + 10^{-6}) \approx 0.0737.
\]
若 mini-batch 中两个类别各占一半样本,则 $\beta_1 = \beta_2 = 0.5$,最终
\[
    \mathcal{L}_{\text{CEL}} = 0.5 \times 0 + 0.5 \times 0.0737 = 0.0369.
\]
这验证了:当某类散度超阈值时,CEL 立即产生惩罚,推动编码器与注意力进行调整。

\section{注意力与门控如何协同降低条件熵}

\begin{itemize}[leftmargin=2.3em]
    \item \textbf{柔性权重}:注意力提供样本级可微权重,使模型自动识别“可信”样本;
    \item \textbf{门控过滤}:门控 $\sigma(W_U \tilde{z})$ 抑制噪声特征,确保 $\bm{v}$ 的重要维度不被干扰;
    \item \textbf{可逆性降低}:通过减小 $v_c$,我们约束了同类 smashed data 的方差,相当于减小了条件熵 $H(Z\mid Y=c)$,攻击者很难从 $z$ 反推具体输入;
    \item \textbf{端到端训练}:整个过程嵌入主训练循环,梯度自动传回编码器与注意力模块,不需额外阶段。
\end{itemize}

\section{关键超参数与直觉}

虽然本报告不讨论训练细节,但为了帮助理解机制,列出影响注意力工作方式的几个重要超参数:
\begin{itemize}[leftmargin=2.3em]
    \item 隐层维度 $h$:越大表示能力越强,但计算量也越大;
    \item 阈值 $\tau$:控制惩罚触发点,过大导致正则过强,过小可能形同虚设;
    \item 平滑项 $\varepsilon$:通常取 $10^{-6}$ 左右;
    \item 缩放系数(若有)$\gamma$:决定正则对总损失的影响力度。
\end{itemize}
这些参数在实际工程中需要根据分类性能与隐私需求综合平衡。这里只需理解它们在数学上的作用——调整注意力对样本选择的敏感度,以及 CEL 惩罚的强度。

\section{总结}

我们以流水线的形式展示了 \texttt{gated-att} 中门控注意力条件熵正则的完整计算过程:
\begin{enumerate}[leftmargin=2.3em]
    \item smashed data 作为输入;
    \item 通过 LayerNorm 规范化后,经双线性层生成 $\bm{v}$ 与 $\bm{u}$;
    \item 利用门控得到 $\bm{s}$,再经线性层和 softmax 得到注意力权重;
    \item 权重决定哪些样本在加权均值、加权方差中更重要;
    \item 超过阈值的方差通过 log 门控转化为正则惩罚;
    \item 正则项的梯度同时作用于编码器与注意力模块,以收紧 smashed data。
\end{enumerate}
通过这种机制,我们达到了用端到端可学习的方式降低条件熵的目的,为协同推理系统提供了实用的隐私防护手段。

\section*{参考文献}
\begin{thebibliography}{9}
    \bibitem{ilse2018attention}
    Ilse, M., Tomczak, J. M., \& Welling, M.\\
    \newblock Attention-based Deep Multiple Instance Learning.\\
    \newblock In \emph{Proceedings of the International Conference on Machine Learning (ICML)}, 2018.
\end{thebibliography}

\end{document}


