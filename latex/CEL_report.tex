\documentclass[12pt]{article}

\usepackage[UTF8]{ctex}            % 中文支持,XeLaTeX 编译
\usepackage{geometry}
\usepackage{amsmath, amssymb, amsfonts, bm}
\usepackage{mathtools}
\usepackage{physics}
\usepackage{booktabs}
\usepackage{array}
\usepackage{multirow}
\usepackage{caption}
\usepackage{subcaption}
\usepackage{graphicx}
\usepackage{tikz}
\usepackage{pgfplots}
\usepackage{algorithm}
\usepackage{algpseudocode}
\usepackage{enumitem}
\usepackage{hyperref}
\usepackage{cleveref}
\usepackage{xcolor}
\usepackage{float}

\geometry{a4paper, margin=2.5cm}
\setlength{\parskip}{0.75em}
\setlength{\parindent}{2em}

\title{条件熵正则项(CEL)在 \texttt{CEM-main} 与 \texttt{gated-att} 中的计算机制详解}
\author{ }
\date{\today}

\begin{document}

\maketitle

\tableofcontents

\newpage

\section{引言与整体目标}

本文面向 \texttt{CEM-main} 与 \texttt{gated-att} 两个项目,剖析其在协同推理(split inference)场景下用于防御模型反演攻击的条件熵正则项(Conditional Entropy Loss, CEL)的全流程计算方式。为方便比较与复现,我们从“输入数据形态、统计量构造、损失公式、梯度回传策略”四个角度细致梳理两种方法的异同,并给出一个可手算的小规模示例用于说明。

在整个系统中,客户端编码器 $f_{\bm{\theta}}$ 将输入图像 $x$ 映射为被称为 smashed data 的中间表示 $z = f_{\bm{\theta}}(x)$。服务器端根据 $z$ 接续推理,但攻击者也可能利用 $z$ 重建出原始图像,构成隐私威胁。CEL 的设计目标是:\textbf{约束同类 smashed data 在特征空间中尽量紧凑},降低攻击者从 $z$ 重建出原始输入的可行性。

\section{统一符号约定}

\begin{itemize}[leftmargin=2.2em]
    \item 输入样本 $(x, y)$,其中 $x \in \mathbb{R}^{h \times w \times c}$,标签 $y \in \{1,2,\dots,C\}$。
    \item 客户端编码器 $f_{\bm{\theta}}: \mathbb{R}^{h \times w \times c} \rightarrow \mathbb{R}^d$,输出 $z = f_{\bm{\theta}}(x)$。
    \item 当前 batch 的 smashed data 记为 $Z_B = \{z_i\}_{i=1}^{|B|}$,标签为 $Y_B = \{y_i\}_{i=1}^{|B|}$。
    \item 全局训练集中类别 $c$ 的样本集合记为 $\mathcal{D}^{(c)}$,在某轮训练得到的 smashed data 全量集合为 $Z_{\text{all}}$。
    \item 分类损失为 $\mathcal{L}_{\text{CE}}$,CEL 记为 $\mathcal{L}_{\text{CEL}}$,总损失 $\mathcal{L} = \mathcal{L}_{\text{CE}} + \lambda \mathcal{L}_{\text{CEL}}$。
    \item 超参数:簇数 $K$、log-entropy 阈值 $\tau$、噪声方差上界 $\sigma^2$、warm-up 轮数 $T_{\text{warm}}$ 等。
\end{itemize}

\section{\texttt{CEM-main} 中的 CEL 计算流程}

\subsection{总体思路}

\texttt{CEM-main} 以“\textbf{离线聚类 + 方差门控}”为核心。即在每个 epoch 结束后,收集所有 smashed data,按类别执行 $K$-means 或混合高斯(GMM)聚类,借助簇方差衡量 smashed data 的分布紧致度。下一轮训练时,批内的 smashed data 通过聚类结果获得其类内方差估计,从而形成 CEL。

\subsection{原始数据与统计量构建}

\paragraph{步骤 1:全量 smashed data 缓存。}

设当前 epoch 中遍历完所有 mini-batch 后得到的 smashed data 集合为
\[
    Z_{\text{all}} = \{(z_i, y_i)\}_{i=1}^{N}, \quad z_i = f_{\bm{\theta}}(x_i),
\]
其中 $N$ 为当轮训练样本总数。数据被分门别类地组织为
\[
    Z_{\text{all}}^{(c)} = \{ z_i \mid y_i = c \}.
\]

\paragraph{步骤 2:按类聚类。}

对每个类别 $c$,执行带暖启动的 $K$-means:

\begin{align*}
    &\text{初始化簇中心 } \{\mu_{c,k}^{(0)}\}_{k=1}^{K} \text{(可使用上轮结果)。} \\
    &\textbf{repeat} \\
    &\qquad \text{分配:}S_{c,k}^{(t)} = \big\{ z \in Z_{\text{all}}^{(c)} \mid k = \arg\min_{k'} \lVert z - \mu_{c,k'}^{(t)} \rVert_2^2 \big\} \\
    &\qquad \text{更新:}\mu_{c,k}^{(t+1)} = \frac{1}{|S_{c,k}^{(t)}|} \sum_{z \in S_{c,k}^{(t)}} z \\
    &\textbf{until} \text{ 收敛或达到最大迭代次数}
\end{align*}

得到簇中心 $\mu_{c,k}$,进一步估计每个簇的方差与协方差:
\begin{align}
    v_{c,k} &= \frac{1}{|S_{c,k}|} \sum_{z \in S_{c,k}} \lVert z - \mu_{c,k} \rVert_2^2, \\
    \Sigma_{c,k} &\approx \operatorname{diag}\!\left( \frac{1}{|S_{c,k}|}\sum_{z \in S_{c,k}} (z - \mu_{c,k})(z - \mu_{c,k})^\top \right).
\end{align}
同时记录簇权重 $\pi_{c,k} = \frac{|S_{c,k}|}{|Z_{\text{all}}^{(c)}|}$。

若采用 GMM,则借助 EM 算法更新高斯混合参数 $(\pi_{c,k}, \mu_{c,k}, \Sigma_{c,k})$。无论 $K$-means 还是 GMM,最后都得到一组可复用的统计量:
\[
    \mathcal{S}_c = \{(\pi_{c,k}, \mu_{c,k}, \Sigma_{c,k})\}_{k=1}^{K}.
\]

\subsection{训练迭代中的 CEL 估计}

在下一轮训练的任意一个 mini-batch 中,考虑类别 $c$ 的子集
\[
    Z_B^{(c)} = \{ z_i \in Z_{B} \mid y_i = c \}.
\]

\paragraph{步骤 1:簇匹配。} 对每个 $z \in Z_B^{(c)}$,根据上节中的簇中心寻找最近簇:
\[
    k^\star(z) = \arg\min_{k} \lVert z - \mu_{c,k} \rVert_2.
\]
\paragraph{步骤 2:类内方差估计。}

利用批内样本与簇中心的欧氏距离构造方差期望:
\begin{equation}
    \hat{v}_c = \sum_{k=1}^{K} \pi_{c,k} \left( \frac{1}{|S_{c,k}^{(B)}|} \sum_{z \in S_{c,k}^{(B)}} \lVert z - \mu_{c,k} \rVert_2^2 \right),
\end{equation}
其中 $S_{c,k}^{(B)} = \{ z \in Z_B^{(c)} \mid k^\star(z) = k \}$,若为空集则忽略该簇。该量可被视为真实类条件分布 $p(z|y=c)$ 的经验二阶矩近似。

\paragraph{步骤 3:门控(Log-entropy)变换。}

设定阈值 $\tau$ 与平滑项 $\varepsilon$,定义
\begin{equation}
    \mathcal{R}_{\text{log}}(c) = \max\left(0, \log(\hat{v}_c + \varepsilon) - \log(\tau + \varepsilon) \right),
\end{equation}
或线性形式 $\mathcal{R}_{\text{lin}}(c) = \hat{v}_c$。最终 CEL 为
\begin{equation}
    \mathcal{L}_{\text{CEL}} = \sum_{c=1}^{C} \beta_c \, \mathcal{R}(c),
\end{equation}
其中 $\beta_c = \frac{|Z_B^{(c)}|}{|B|}$ 是当前 batch 的类别占比。

\subsection{梯度与优化}

总损失 $\mathcal{L} = \mathcal{L}_{\text{CE}} + \lambda \mathcal{L}_{\text{CEL}}$。为强调 CEL 仅用于约束客户端编码器,实践中采取“先对 $\mathcal{L}_{\text{CEL}}$ 反向传播捕获梯度,再清零梯度、对 $\mathcal{L}_{\text{CE}}$ 反传并与前者合并”的策略,使得 CEL 的梯度只作用于 $\bm{\theta}$。

\subsection{特性小结}

\begin{itemize}[leftmargin=2.2em]
    \item 优点:统计更精细,能够捕捉多峰分布;对噪声注入、防御策略兼容性好。
    \item 缺点:需缓存全量 smashed data;聚类耗时且易受离群点影响;簇数 $K$ 需调参。
\end{itemize}

\section{\texttt{gated-att} 中的 CEL 计算流程}

\subsection{总体思路}

\texttt{gated-att} 直接把“\textbf{可微注意力权重}”作为条件熵 surrogate,不再需要离线聚类。每个 batch 内,使用门控注意力对同类 smashed data 做加权聚合,计算加权方差,借此定义 CEL。注意力参数随训练迭代一同更新,实现端到端学习。

\subsection{原始数据与批内处理}

对于当前 batch $B$,考虑类别 $c$ 的样本集合 $Z_B^{(c)}$,其大小记为 $m_c = |Z_B^{(c)}|$。引入门控注意力模块
\[
    \mathcal{A}_{\bm{\phi}}: \mathbb{R}^d \rightarrow (0,1), \quad \bm{\phi} = \{W_V, W_U, \bm{w}\},
\]
其中
\begin{align}
    \bm{v}_i &= \tanh(W_V z_i), \\
    \bm{u}_i &= \sigma(W_U z_i), \\
    s_i &= \bm{w}^\top (\bm{v}_i \odot \bm{u}_i), \\
    \alpha_i &= \frac{\exp(s_i)}{\sum_{j=1}^{m_c} \exp(s_j)}.
\end{align}
$\sigma(\cdot)$ 为 Sigmoid,$\odot$ 为逐元素乘法。

\subsection{CEL 计算公式}

\paragraph{加权统计量。}
\begin{align}
    \bar{z}_c &= \sum_{i=1}^{m_c} \alpha_i z_i, \\
    v_c &= \sum_{i=1}^{m_c} \alpha_i \lVert z_i - \bar{z}_c \rVert_2^2,
\end{align}
其中 $\alpha_i$ 满足 $\sum_i \alpha_i = 1$。

\paragraph{门控阈值。}

同样提供线性或 log-entropy 两种形式:
\begin{align}
    \mathcal{R}_{\text{lin}}(c) &= v_c, \\
    \mathcal{R}_{\text{log}}(c) &= \max\left(0, \log(v_c + \varepsilon) - \log(\tau + \varepsilon)\right),
\end{align}
最终 CEL 为
\begin{equation}
    \mathcal{L}_{\text{CEL}} = \sum_{c=1}^{C} \beta_c \, \mathcal{R}(c), \qquad \beta_c = \frac{m_c}{|B|}.
\end{equation}

\paragraph{Warm-up 策略。}

为了避免初期注意力未收敛导致训练震荡,引入 warm-up 轮数 $T_{\text{warm}}$:在 $t \leq T_{\text{warm}}$ 的 epoch 内 $\mathcal{L}_{\text{CEL}}$ 被禁用,仅训练分类损失;在 $t > T_{\text{warm}}$ 时才逐渐打开 CEL,并按缩放系数 $\gamma$ 调节其贡献:
\begin{equation}
    \mathcal{L} = \mathcal{L}_{\text{CE}} + \lambda \gamma \mathcal{L}_{\text{CEL}}, \quad 0 < \gamma \leq 1.
\end{equation}

\subsection{梯度与优化}

由于 $\mathcal{L}_{\text{CEL}}$ 完全可微,其梯度同时作用于编码器参数 $\bm{\theta}$ 与注意力参数 $\bm{\phi}$。当首次启用 CEL 时,将 $\bm{\phi}$ 加入优化器的参数组,其学习率、动量等超参与主干一致或单独设定。

\subsection{特性小结}

\begin{itemize}[leftmargin=2.2em]
    \item 优点:端到端可微,无需存储全量 smashed data;对动态分布变化响应快。
    \item 缺点:注意力参数需额外调整;对 batch 内样本量较少的类别方差估计可能有偏,需要通过 $\varepsilon$ 或正则项稳定。
\end{itemize}

\section{步骤对比与时间/空间开销分析}

\begin{table}[H]
    \centering
    \caption{两种方案的关键差异一览}
    \begin{tabular}{p{4cm} p{5.5cm} p{5.5cm}}
        \toprule
        方面 & \texttt{CEM-main}(聚类法) & \texttt{gated-att}(注意力法) \\
        \midrule
        原始数据 & 全量 smashed data & 当前 batch smashed data \\
        统计方式 & $K$-means / GMM 离线估计中心、方差 & 门控注意力实时计算加权均值与方差 \\
        是否缓存 & 需缓存所有 smashed data & 无需缓存 \\
        额外参数 & 无,纯统计 & 注意力参数 $(W_V, W_U, \bm{w})$ \\
        CEL 形式 & $\sum_c \pi_{c,k} \Vert z - \mu_{c,k} \Vert^2$ 的 log/线性门控 & 加权方差 $v_c$ 的 log/线性门控 \\
        训练阶段 & 每轮结束后聚类,下一轮使用 & 每个 batch 实时计算 \\
        梯度流向 & 仅回传到编码器 $\bm{\theta}$ & 同时更新编码器 $\bm{\theta}$ 和注意力 $\bm{\phi}$ \\
        超参敏感性 & 对簇数 $K$、初始化敏感 & 对 warm-up、注意力维度敏感 \\
        算法复杂度 & 聚类 $O(NKd)$,内存 $O(Nd)$ & 每 batch $O(|B|d)$,内存 $O(|B|d)$ \\
        \bottomrule
    \end{tabular}
\end{table}

\section{手工示例:三类二维特征的完整推导}

为了更直观展示两种 CEL 的计算过程,下面构造一个小规模示例。设编码器输出为二维向量,存在三个类别,每类 2 条 smashed data:

\begin{align*}
    Z^{(1)} &= \{ (0.0, 0.0), (0.2, 0.1) \}, \\
    Z^{(2)} &= \{ (1.0, 1.0), (1.2, 1.1) \}, \\
    Z^{(3)} &= \{ (0.9, -0.9), (1.1, -1.0) \}.
\end{align*}

考虑 batch 恰好包含全部 6 条样本,label 均匀。取阈值 $\tau = 0.05$,平滑 $\varepsilon = 10^{-6}$,聚类簇数 $K=1$。

\subsection{\texttt{CEM-main} 计算步骤}

\paragraph{离线聚类。}

由于每类仅两点,且 $K=1$,聚类结果就是各自的样本均值:
\begin{align*}
    \mu_{1,1} &= (0.1, 0.05), \\
    \mu_{2,1} &= (1.1, 1.05), \\
    \mu_{3,1} &= (1.0, -0.95).
\end{align*}
簇权重 $\pi_{c,1} = 1$。方差估计:
\begin{align*}
    v_{1,1} &= \frac{1}{2} \big( \|(0,0) - \mu_{1,1}\|_2^2 + \|(0.2, 0.1) - \mu_{1,1}\|_2^2 \big) \\
            &= \frac{1}{2} \left( 0.125^2 + 0.125^2 \right) \approx 0.03125, \\
    v_{2,1} &\approx 0.0100, \\
    v_{3,1} &\approx 0.0100.
\end{align*}

\paragraph{批内方差。}

由于 batch 与离线集合一致,再次计算也得到相同的 $\hat{v}_c = v_{c,1}$。

\paragraph{CEL 计算。}

采用 log-entropy:
\begin{align*}
    \mathcal{R}_{\text{log}}(1) &= \max\bigl(0, \log(0.03125+\varepsilon) - \log(0.05+\varepsilon) \bigr) = 0, \\
    \mathcal{R}_{\text{log}}(2) &= 0, \\
    \mathcal{R}_{\text{log}}(3) &= 0.
\end{align*}
若使用线性形式,则
\[
    \mathcal{L}_{\text{CEL}}^{\text{lin}} = \sum_{c=1}^3 \frac{2}{6} v_{c,1} \approx 0.0171.
\]

\paragraph{梯度方向。}

对类别 1 的第一个样本 $z_1 = (0,0)$,梯度为
\[
    \pdv{ \mathcal{L}_{\text{CEL}}^{\text{lin}} }{ z_1 } = \frac{2}{6} \cdot 2 (z_1 - \mu_{1,1}) = \frac{2}{3} (-0.1, -0.05).
\]
说明编码器参数将被调整,使类别 1 的样本更接近其中心。

\subsection{\texttt{gated-att} 计算步骤}

\paragraph{注意力权重(示例取等权)。}

假设经过若干轮训练,注意力网络已学习到对称权重,即 $\alpha_i = \frac{1}{2}$。则
\[
    \bar{z}_c = \frac{1}{2} z^{(c)}_1 + \frac{1}{2} z^{(c)}_2 = \mu_{c,1}.
\]

\paragraph{加权方差。}

\begin{align*}
    v_1 &= \frac{1}{2}\| (0,0) - \mu_{1,1} \|_2^2 + \frac{1}{2}\| (0.2,0.1) - \mu_{1,1} \|_2^2 \approx 0.03125, \\
    v_2 &\approx 0.0100, \quad v_3 \approx 0.0100.
\end{align*}

\paragraph{CEL。}

与聚类法一致,线性形式 $\mathcal{L}_{\text{CEL}}^{\text{lin}} \approx 0.0171$。

\paragraph{注意力梯度。}

若实际训练中 $\alpha_i$ 不是 $\frac{1}{2}$,CEL 的梯度将推动注意力网络向“抑制远离中心的样本”方向更新。例如若类别 1 中注意力权重 $\alpha_1 = 0.7, \alpha_2 = 0.3$,则
\[
    \bar{z}_1 = 0.7\,z_1 + 0.3\,z_2, \quad v_1 = 0.7\|z_1 - \bar{z}_1\|_2^2 + 0.3\|z_2 - \bar{z}_1\|_2^2.
\]
若 $z_1$ 离中心更远,梯度会增大注意力网络对 $z_2$ 的权重,以减小 $v_1$。

\subsection{示例结论}

在上述简单示例中,两种方法给出的线性 CEL 数值相同。但在更复杂数据上,\texttt{CEM-main} 依赖历史聚类,可能捕捉到 batch 之外的全局结构;\texttt{gated-att} 则利用注意力实时对 batch 内样本加权。前者对批间一致性敏感,后者对注意力网络的训练稳定性敏感。

\section{实践建议与调参要点}

\subsection{\texttt{CEM-main}}
\begin{itemize}[leftmargin=2.2em]
    \item \textbf{簇数 $K$}:可按类别 smashed data 的多峰程度设定。经验上 CIFAR-10 使用 $K=5$--$10$。
    \item \textbf{聚类频率}:可每轮聚类一次,也可每若干轮聚类以降低开销。
    \item \textbf{阈值 $\tau$}:与噪声注入强度 $\sigma^2$ 协同调节,过小会导致梯度爆炸。
\end{itemize}

\subsection{\texttt{gated-att}}
\begin{itemize}[leftmargin=2.2em]
    \item \textbf{Warm-up 轮数}:建议至少 5--10 轮,确保分类性能先收敛。
    \item \textbf{注意力隐层维度}:默认取 $d/4$ 或 128,过大易过拟合,过小表达不足。
    \item \textbf{缩放系数 $\gamma$}:可从 $0.1$ 起逐渐增大,观察对准确率与重建质量的影响。
\end{itemize}

\section{结论}

本文以数学视角详细梳理了 \texttt{CEM-main} 与 \texttt{gated-att} 在条件熵正则上的实现差异。二者在目标上一致,均致力于压缩 smashed data 的类内可分散度;但在具体实现中,一个依赖离线聚类、另一个借助门控注意力。根据实际场景的计算资源、数据规模、对实时性的要求,可灵活选择或进一步结合两者的优点(例如利用注意力初始化聚类中心,或在注意力损失中引入历史统计的先验项)。

\section*{参考阅读}
\begin{itemize}[leftmargin=2.2em]
    \item Xia et al., ``Theoretical Insights in Model Inversion Robustness and Conditional Entropy Maximization for Collaborative Inference Systems,'' CVPR 2025.
    \item Ilse et al., ``Attention-based Deep Multiple Instance Learning,'' ICML 2018.
    \item 常见模型反演攻击综述与开源实现(\texttt{GMI}, \texttt{BiDO} 等)。
\end{itemize}

\end{document}
