\documentclass[12pt]{article}
\usepackage{CJKutf8}
\usepackage{amsmath, amssymb, amsfonts}
\usepackage{geometry}
\usepackage{graphicx}
\usepackage{booktabs}
\usepackage{listings}
\usepackage{enumitem}
\usepackage{hyperref}
\geometry{a4paper, margin=1in}
\hypersetup{
    colorlinks=true,
    linkcolor=blue,
    urlcolor=cyan
}
\setlength{\parskip}{0.6em}
\setlength{\parindent}{2em}
\title{Slot + Gated Cross 条件熵代理框架解析报告}
\author{自动生成}
\date{\today}

\begin{document}
\begin{CJK*}{UTF8}{gbsn}
\maketitle

\begin{abstract}
本文面向项目根目录中的 \texttt{SlotCrossAttentionCEM} 模块,系统梳理 ``slot + gated cross'' 条件熵(Conditional Entropy Minimization, CEM)代理的设计思想、数学建模与训练流程。报告首先回顾 CEM 在隐私防护/表示学习中的目标,再详细拆解 Slot Attention、门控交叉注意力、混合槽统计与门控策略等关键组件,并解释这些设计如何替代原有的高斯混合模型(GMM)近似器来稳定地估计条件熵损失。最后我们分析该模块在训练循环中的反向传播路径、超参数选择与调试建议。
\end{abstract}

\tableofcontents

\section{背景:条件熵最小化(CEM)}

\subsection{目标函数}
在会员推理防御与判别式表示学习任务中,我们希望编码器产出的特征 $Z$ 尽可能与类别标签 $Y$ 对应且类内紧凑。条件熵
\begin{equation}
    H(Z \mid Y) = \sum_{c} p(c) \, H(Z \mid Y=c)
\end{equation}
刻画了在给定类别情况下特征分布的不确定性。若我们能最小化 $H(Z \mid Y)$,就能抑制类内方差、削弱隐私攻击对样本细节的重构能力。

在实践中不可直接获得 $p(z \mid y)$,常见近似是假设类条件分布近似高斯,此时
\begin{equation}
    H(Z \mid Y=c) \approx \frac{1}{2} \log \det \bigl( 2\pi e\, \Sigma_c \bigr),
\end{equation}
其中 $\Sigma_c$ 为类别 $c$ 的协方差矩阵。传统方法通常使用 KMeans/GMM 来拟合类别子簇,再估计方差或熵。然而在高维空间、批大小有限或者分布多模态时,这类方法容易数值不稳定或对初始化敏感。

\subsection{设计动机}
``slot + gated cross'' 框架的设计出发点在于:
\begin{itemize}[leftmargin=2em]
    \item 用 Slot Attention 自适应地捕获类内多个潜在子模式(代替 GMM 的硬聚类中心)。
    \item 通过 Flamingo 风格的门控交叉注意力,让每个样本在共享槽(slot)上下文中提取增强特征,获得更稳定的局部统计量。
    \item 在方差估计上引入多级门控(维度门、SNR 门、槽质量门、类级门)与软阈处理,以抑制噪声方差并控制梯度爆炸。
\end{itemize}

\section{框架总览}

整体流程按类别拆分为三个阶段:
\begin{enumerate}[leftmargin=2em]
    \item \textbf{Slot Attention 汇聚}:将同类样本嵌入 $\{x_m\}_{m=1}^{M}$ 看成序列,经过多次竞争-更新获得一组共享槽 $\{s_k\}_{k=1}^{S}$。
    \item \textbf{门控交叉注意力增强}:以样本特征为查询、槽为键值,通过多头交叉注意力并配合门控残差,产生增强后的特征 $\{\tilde{x}_m\}$。
    \item \textbf{混合槽统计 \& 条件熵 surrogate}:基于归一化相似度推导软责任 $r_{mk}$,进而得到槽均值、槽方差,并在多级门控与阈值平滑下组合成类级的对数方差指标。
\end{enumerate}

\section{组件详解}

\subsection{Slot Attention 机制}
Slot Attention 模块位于 \texttt{model\_training\_paral\_pruning.py} 第 35--87 行。输入为形状 $[B, N, D]$ 的序列,其中当前用法是 $B=1$、$N=M$(类内样本数)。每次迭代步骤如下:
\begin{align}
    k_n &= W_k \, \mathrm{LN}(x_n), \quad v_n = W_v \, \mathrm{LN}(x_n), \\
    q_k &= W_q \, \mathrm{LN}(s_k^{(t-1)}) / \sqrt{d}, \\
    a_{nk} &= \mathrm{softmax}_k\!\left(\frac{k_n^\top q_k}{\tau}\right), \qquad \tau = (d)^{1/4}, \\
    u_k &= \sum_{n} a_{nk} v_n, \\
    s_k^{(t)} &= \mathrm{GRU}\bigl(u_k, s_k^{(t-1)}\bigr) + \mathrm{MLP}\bigl(\mathrm{LN}(\mathrm{GRU}(\cdot))\bigr).
\end{align}
相较原始论文,这里额外引入 $\tau = d^{1/4}$ 的温度来缓和对数似然值的尺度,避免梯度过大;同时层归一化保证不同批次的数值稳定。

\subsection{门控交叉注意力}
交叉注意力模块定义在第 89--131 行。给定单个样本的特征 $x_m$(视为 $[1,D]$),以及共享槽 $\{s_k\}$,流程为:
\begin{align}
    q &= \mathrm{LN}_q(x_m), \quad K = \mathrm{LN}_{kv}(S), \quad V = \mathrm{LN}_{kv}(S), \\
    \alpha &= \mathrm{softmax}\left(\frac{q K^\top}{\sqrt{d}}\right), \\
    h &= \alpha V, \qquad o = W_o h, \\
    y &= x_m + \tanh(\alpha_{\text{xattn}}) \cdot o, \\
    y &= y + \tanh(\alpha_{\text{ffn}}) \cdot \mathrm{FFN}(\mathrm{LN}_{ff}(y)).
\end{align}
其中 $\alpha_{\text{xattn}}$、$\alpha_{\text{ffn}}$ 是可学习的门控系数,初始值较小使得网络可以平滑地从恒等映射过渡到使用注意力。该结构借鉴了 Flamingo 模型的 \emph{Gated Cross-Attn Dense} 设计,使得注意力和前馈子层的贡献可被动态调节。

\subsection{混合槽统计与条件熵 surrogate}
位于第 179--275 行的是核心的 CEM surrogate 计算。对每个类别 $c$:
\begin{enumerate}[leftmargin=2em]
    \item \textbf{样本归一化}:对特征做 LayerNorm,缓解尺度差异导致的偏置。
    \item \textbf{槽责任分配}:采用余弦相似度
    \begin{equation}
        s_{mk} = \frac{\langle \hat{x}_m, \hat{s}_k \rangle}{\|\hat{x}_m\| \cdot \|\hat{s}_k\|}, \qquad
        r_{mk} = \frac{\exp(\beta s_{mk})}{\sum_{k'} \exp(\beta s_{mk'})},
    \end{equation}
    其中 $\beta$(代码中的 \texttt{assign\_temp})是可学习的温度,控制责任分布的尖锐程度。槽质量 $w_k = \sum_m r_{mk}$ 用来度量该槽代表的样本量。
    \item \textbf{槽均值与槽方差}:
    \begin{align}
        \mu_k &= \frac{1}{w_k} \sum_m r_{mk} \,\tilde{x}_m, \\
        \sigma_{k,d}^2 &= \frac{1}{w_k} \sum_m r_{mk} \bigl(\tilde{x}_{m,d} - \mu_{k,d}\bigr)^2,
    \end{align}
    并通过 $\log \sigma_{k,d}^2$ 来后续构建熵 surrogate。
    \item \textbf{维度门控}:对每个槽的 $\log \sigma^2$ 先经 LayerNorm,再经过带 Sigmoid 输出的 MLP(第 144--152 行)得到软门 $g_{k,d}^{(\text{soft})}$,用于抑制噪声方差;同时计算信噪比
    \begin{equation}
        \mathrm{SNR}_{k,d} = \frac{\sigma_{k,d}^2}{\mu_{k,d}^2 + \varepsilon},
    \end{equation}
    并通过 $\mathrm{sigmoid}(\alpha_{\text{snr}}(\mathrm{SNR}-\tau_{\text{snr}}))$ 形成近似硬门 $g_{k,d}^{(\text{hard})}$。
    \item \textbf{软阈平滑}:相较于直接使用 $\mathrm{ReLU}(\log \sigma^2 - \log \sigma^2_{\text{thr}})$,这里改为
    \begin{equation}
        \phi_{k,d} = \frac{1}{\beta_{\text{sp}}} \log\left(1 + \exp\bigl(\beta_{\text{sp}} (\log\sigma_{k,d}^2 - \log\sigma_{\text{thr}}^2 - m)\bigr)\right),
    \end{equation}
    对应代码中的 \texttt{softplus\_beta} 与 \texttt{margin\_m}。该设计可以在阈值附近提供平滑梯度。
    \item \textbf{槽权重门控}:槽质量经幂次 $\gamma$(\texttt{slot\_power})强化后归一化:
    \begin{equation}
        \tilde{w}_k = \frac{(w_k / M)^\gamma}{\sum_{k'} (w_{k'} / M)^\gamma},
    \end{equation}
    使得代表性较强的槽贡献更大。
    \item \textbf{类级门控与聚合}:将维度门、SNR 门、阈平滑项相乘
    \begin{equation}
        \psi_{k,d} = g_{k,d}^{(\text{soft})} \cdot g_{k,d}^{(\text{hard})} \cdot \phi_{k,d},
    \end{equation}
    然后先按槽权重求和,再对维度求平均得到类级 surrogate $\hat{L}_c$:
    \begin{equation}
        \hat{L}_c = \frac{1}{D} \sum_{d} \sum_{k} \tilde{w}_k \, \psi_{k,d}.
    \end{equation}
    为了避免小批次导致估计偏差,再用类别样本占比 $p_c = M / B$ 进入 Sigmoid 门
    \begin{equation}
        g_c = \sigma\bigl(a (p_c - b)\bigr),
    \end{equation}
    (\texttt{class\_gate\_a}, \texttt{class\_gate\_b}),最终类级贡献为 $g_c \cdot \hat{L}_c$。
\end{enumerate}

对所有类别求加权平均(权重即 $p_c$)即可得到
\begin{equation}
    \mathrm{rob\_loss} = \sum_c p_c \, g_c \, \hat{L}_c, \qquad
    \mathrm{intra\_mse} = \sum_c p_c \, \mathrm{MSE}_c,
\end{equation}
其中 $\mathrm{MSE}_c$ 仅用于日志监控。

\section{门控策略与稳定性设计}

\subsection{早期关断机制}
在训练初期模型尚未收敛时,方差估计极易震荡。代码通过维护滑动平均的门值,并在以下任一条件成立时将 $\mathrm{rob\_loss}$ 强制置零:
\begin{itemize}[leftmargin=2em]
    \item 模块调用次数 $\leq$ \texttt{early\_shut\_steps}。
    \item SNR 硬门平均值大于 \texttt{early\_hard\_thresh}。
    \item 软门平均值大于 \texttt{early\_gate\_thresh}。
\end{itemize}
这确保早期梯度不会把编码器推向坏的局部最优,同时仍保留计算图以便后续渐进启用。

\subsection{可学习温度与阈值}
多处门控参数都是可学习的标量,允许网络在训练过程中自动调节:
\begin{itemize}[leftmargin=2em]
    \item $\beta$:责任分布温度,控制槽的“硬”程度。
    \item $\gamma$:槽质量幂,决定是否强调主槽。
    \item SNR 阈值与斜率 $\tau_{\text{snr}}, \alpha_{\text{snr}}$:自动判别低信噪比维度。
    \item Softplus 斜率与边际 $(\beta_{\text{sp}}, m)$:稳定阈值附近的梯度。
    \item 类级门 $(a, b)$:基于批量样本比例调节贡献,避免样本数极小的类别主导损失。
\end{itemize}

\section{与 CEM 训练流程的结合}

\subsection{前向阶段}
主训练循环在 \texttt{train\_target\_step} 中实现(第 1240--1501 行)。当满足
\begin{equation*}
    \texttt{use\_attention\_cem} \land \lnot \texttt{random\_ini\_centers} \land \lambda > 0 \land \texttt{epoch} > \texttt{warmup}
\end{equation*}
时,会将特征展平并送入注意力版 CEM,得到 $\mathrm{rob\_loss}$ 与 $\mathrm{intra\_class\_mse}$。随后将 $\mathrm{rob\_loss}$ 乘以缩放系数(默认 $0.25$)以控制梯度量级。

\subsection{反向与梯度合成}
训练器先对 $\mathrm{rob\_loss}$ 做一次反向传播,仅保留编码器和注意力模块的梯度快照,然后清零优化器梯度;接着对分类损失 $L_{\text{CE}}$ 反向,最终以
\begin{equation}
    \nabla_{\theta_f} = \nabla_{\theta_f} L_{\text{CE}} + \lambda \cdot w_{\text{lr}} \, \nabla_{\theta_f} \mathrm{rob\_loss}
\end{equation}
的形式合并梯度,其中 $w_{\text{lr}}$ 是基于学习率调度器的缩放项。注意力模块参数则直接累加两次反向的梯度。该做法相当于执行一次“显式梯度加权”,避免在总损失中直接拼接两个项导致训练初期不稳定。

\section{实现与调优建议}

\subsection{超参数与默认值}
\begin{itemize}[leftmargin=2em]
    \item 槽数量 $S=8$(可根据类内模式复杂度调节)。
    \item Slot Attention 迭代次数 $T=3$,折中考虑效率与表示力。
    \item 责任温度、槽幂、SNR 阈值、Softplus 斜率等均为可学习参数,训练中会自动调整。
    \item 训练初期使用 \texttt{attention\_warmup\_epochs}=3 暂停 CEM 梯度,待编码器初步收敛后再启用。
    \item $\mathrm{rob\_loss}$ 的缩放 \texttt{attention\_loss\_scale}=0.25,可以视情况放大以增强防护,也可减小以保证主任务精度。
\end{itemize}

\subsection{数值稳定性心得}
\begin{itemize}[leftmargin=2em]
    \item 在高维特征上,LayerNorm 与温度缩放是必要的,否则余弦相似度与对数方差容易溢出。
    \item Softplus 边际 $m$ 能防止 $\sigma^2$ 略低于阈值时出现梯度突变,建议与 $\texttt{var\_threshold}$ 联动调节。
    \item 若批次类别数量过少,可适当放宽 \texttt{early\_shut} 条件或降低类级门 $a$,否则 CEM 长时间处于关闭状态。
\end{itemize}

\section{结论}

Slot + Gated Cross 框架以模块化方式重写了 CEM 近似器:Slot Attention 捕获多模态子簇,门控交叉注意力在共享上下文中重投影样本,随后利用多重门控的混合槽统计稳定地估计类内对数方差,从而提供光滑、可控的条件熵 surrogate。通过分阶段反向与梯度合并,该模块实现了与主任务的解耦训练,同时保留对编码器的正则化力度。实务中建议结合 warmup、门控参数监控与梯度缩放来调节防护-精度平衡。

\end{CJK*}
\end{document}
