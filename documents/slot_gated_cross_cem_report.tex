\documentclass[12pt]{article}
\usepackage{CJKutf8}
\usepackage{amsmath, amssymb, amsfonts}
\usepackage{geometry}
\usepackage{graphicx}
\usepackage{booktabs}
\usepackage{listings}
\usepackage{enumitem}
\usepackage{hyperref}
\geometry{a4paper, margin=1in}
\hypersetup{
    colorlinks=true,
    linkcolor=blue,
    urlcolor=cyan
}
\setlength{\parskip}{0.6em}
\setlength{\parindent}{2em}
\title{Slot + Gated Cross 条件熵代理框架易懂手册}
\author{自动生成}
\date{\today}

\begin{document}
\begin{CJK*}{UTF8}{gbsn}
\maketitle

\begin{abstract}
这份文档面向第一次接触项目的读者。我们会用生活化的比喻解释 Slot Attention + Gated Cross Attention 版 CEM 的整套流程:为什么要这么做、代码里每一步在干什么、门控和阈值的作用是什么、训练时怎么与主任务拼在一起。目标是让你读完后,即便不熟悉注意力或条件熵,也能明白模型的运作逻辑,并知道哪些开关可以调、遇到问题该看哪里。
\end{abstract}

\tableofcontents

\section{先说结论:我们想解决什么问题?}
\begin{itemize}[leftmargin=2em]
    \item \textbf{我们希望类内特征更“抱团”}:同一类别的样本在特征空间里越集中,攻击者就越难根据激活重建原图。
    \item \textbf{CEM(条件熵最小化)就是衡量“抱团”程度}:如果类内很分散,条件熵高;如果集中,条件熵低。
    \item \textbf{老办法不好用}:KMeans / GMM 需要选簇数、初始化,特征维高的时候不稳。
    \item \textbf{我们的方法}:先找“代表队长”(Slot Attention),再让每个样本向队长取经(Gated Cross Attention),最后做一套门控统计判断类内是否紧凑。
\end{itemize}

\section{整体流程像是“班干部+分工”}
\begin{enumerate}[leftmargin=2em]
    \item \textbf{阶段一:挑队长(Slot Attention)}\\
    同类样本站成一排,竞争成为几个“队长”。队长会多轮征召、更新,直到每个子群体都有代表。
    \item \textbf{阶段二:跟队长学习(Gated Cross Attention)}\\
    每个样本向所有队长请教,按注意力权重把队长的经验加到自己身上,但有门控,避免一下子改变太多。
    \item \textbf{阶段三:统计班级是否稳定(门控方差 + CEM)}\\
    用余弦相似度算每个样本跟哪个队长最亲,再按权重统计方差。多重门控过滤噪声、突出重点,最后得到“这类是否松散”的分数。
\end{enumerate}

\section{步骤 1:Slot Attention(挑队长)}
\subsection*{1.1 为什么需要它?}
如果直接对全部样本求方差,会把所有细节都记下来,既费算力又不稳定。Slot Attention 用“几个代表”概括整个类,既能捕捉多模态,又能保持计算量可控。

\subsection*{1.2 它具体做什么?}
\begin{enumerate}[leftmargin=2em]
    \item \textbf{预处理}:每个样本先做 LayerNorm,避免谁数字特别大。
    \item \textbf{生成 Key/Value}:把样本投影成 Key(用于和队长比对)和 Value(真正要聚合的信息)。
    \item \textbf{初始化队长}:从一个可学习的高斯分布里随机抽几个初始槽(队长)。初始就有差异,防止卡在一个点。
    \item \textbf{多轮竞争更新}:\\
    每一轮包括:
    \begin{itemize}
        \item 用上一轮的队长向样本发出 Query(“我想找和我最像的同学”),点积后用 softmax 分配权重。
        \item 按权重把样本的 Value 加权和,得到新的输入。
        \item 用 GRU + MLP 更新队长,既吸收新信息,又保留记忆。
    \end{itemize}
    \item \textbf{多轮迭代后输出}:最终每个队长代表一组相似样本,像“子簇”的中心。
\end{enumerate}

\subsection*{1.3 小技巧}
\begin{itemize}[leftmargin=2em]
    \item 点积里有一个温度系数 $\tau = d^{1/4}$:把分数压缩一些,避免 softmax 太尖锐。
    \item LayerNorm 和 GRU\,+\,MLP 残差结构确保训练稳定。
\end{itemize}

\section{步骤 2:Gated Cross Attention(跟队长学习)}
\subsection*{2.1 目的}
有了队长,还要让每个样本从队长那里“听取意见”,得到更稳的表示。这里用的是跨注意力(样本是 Query,队长是 Key/Value)。

\subsection*{2.2 流程}
\begin{enumerate}[leftmargin=2em]
    \item \textbf{对齐尺度}:样本、队长分别做 LayerNorm 和线性映射。
    \item \textbf{算注意力权重}:样本 Query 去和每个队长 Key 点积,softmax 后得到“我最该听哪个队长”的权重。
    \item \textbf{聚合信息}:按权重加权平均队长的 Value,就得到一个增强特征。
    \item \textbf{门控残差融合}:\\
    先把增强特征通过线性层,再乘上 $\tanh(\alpha_{\text{xattn}})$ 加回原特征。$\alpha$ 是可学的,初始很小,就像一开始只允许少量建议进入。\\
    再走一遍前馈网络(FFN),同样用 $\tanh(\alpha_{\text{ffn}})$ 控制力度。
\end{enumerate}

\subsection*{2.3 为什么要门控?}
\begin{itemize}[leftmargin=2em]
    \item 防止模型一开始就“听命于注意力”,导致主任务学不动。
    \item 训练过程中会慢慢加大门值,等网络学会了,就能充分利用注意力。
\end{itemize}

\section{步骤 3:混合槽统计 \& 条件熵判定}
\subsection*{3.1 三个问题}
\begin{enumerate}[leftmargin=2em]
    \item 哪个样本应该归哪个队长?(责任分配)
    \item 每个队长的“队员”分布集中吗?(加权均值/方差)
    \item 哪些方差是真问题,哪些只是噪声?(门控和阈值)
\end{enumerate}

\subsection*{3.2 逐步拆开}
\begin{enumerate}[leftmargin=2em]
    \item \textbf{再做一次 LayerNorm}:确保增强后的样本可比。
    \item \textbf{余弦相似度 + softmax}:让每个样本得到一组对队长的责任权重 $r_{mk}$。温度 $\beta$ 可学习,决定“分配多平均还是更偏向某一个队长”。
    \item \textbf{算加权均值/方差}:像计算带权重的统计量一样,得到每个队长的 $\mu_s$ 和 $\sigma_s^2$。
    \item \textbf{多重门控过滤噪声}:
    \begin{itemize}
        \item \textbf{维度软门}:对 $\log \sigma^2$ 做 LayerNorm + MLP + Sigmoid,判定“这个维度可靠不可靠”。
        \item \textbf{SNR 硬门}:看 $\sigma^2 / (\mu^2+\varepsilon)$,信噪比低的维度被压低。
        \item \textbf{Softplus 阈值}:不用硬 ReLU,而是用平滑的 Softplus 函数,让阈值附近也有梯度。
    \end{itemize}
    \item \textbf{槽权重}:看每个队长有多少队员(slot mass),对权重做幂次放大,保证“人多的队长”更有话语权。
    \item \textbf{类级聚合}:把上述门控乘起来,先按队长权重求和,再对所有维度取平均,得到类级指标 $L_c$。
    \item \textbf{类级门}:如果某类在当前 batch 里样本很少,就用 Sigmoid 门把它的贡献压小,防止统计误判。
\end{enumerate}

\subsection*{3.3 最终输出}
\begin{itemize}[leftmargin=2em]
    \item \textbf{rob\_loss}:所有类的 $L_c$ 按样本占比加权平均,就得到正则损失。
    \item \textbf{intra\_mse}:同时记录类内 MSE,方便看类内是否在收缩(只是日志,不参与梯度)。
\end{itemize}

\section{训练里怎么用?}
\subsection*{4.1 触发条件}
\begin{itemize}[leftmargin=2em]
    \item 不随机初始化聚类中心、$\lambda > 0$、当前 epoch 超过 warmup(默认 3)。
    \item 首次使用时实例化模块,并把参数加入优化器。
\end{itemize}

\subsection*{4.2 梯度处理}
\begin{enumerate}[leftmargin=2em]
    \item 先对 rob\_loss 反向,保存编码器和注意力模块的梯度。
    \item 清梯度,再对主任务交叉熵反向。
    \item 把 rob\_loss 的梯度按学习率等比例缩放后加回编码器参数里。
    \item 注意力模块的梯度直接相加。
\end{enumerate}
这样做是为了“显式注入”正则的梯度,而不是把它和主损失简单相加导致训练早期不稳定。

\subsection*{4.3 数值保护}
\begin{itemize}[leftmargin=2em]
    \item rob\_loss 或 intra\_mse 出现 NaN/Inf 会被自动置零。
    \item 早期关断逻辑:如果门值过大或还在 warmup,就返回 0,但是梯度图仍然连着。
\end{itemize}

\section{调参和排障清单}
\subsection*{5.1 必调/常用参数}
\begin{itemize}[leftmargin=2em]
    \item \texttt{num\_slots}:队长数量,默认 8。类内子模式越多可以适当增大。
    \item \texttt{num\_iterations}:Slot Attention 迭代次数,默认 3,更多会更精细但更慢。
    \item \texttt{assign\_temp}、\texttt{slot\_power} 等可学习参数无需手调,训练中会自行调节。
    \item \texttt{attention\_loss\_scale}:rob\_loss 的全局缩放,默认 0.25。若正则力度不够,可以逐步调大。
    \item \texttt{var\_threshold}:阈值越大,越宽容;如果希望更严格,可减小该值。
\end{itemize}

\subsection*{5.2 出问题时看哪里?}
\begin{itemize}[leftmargin=2em]
    \item \textbf{日志关键字}:\texttt{[CEM-GATE]} 打印会显示门平均值、rob\_loss。门值太高意味着大部分维度都被判为“不可靠”,可能需要调整阈值或 warmup。
    \item \textbf{NaN}:检查输入特征是否有 Inf/NaN,或者 Softplus 参数是否过大。
    \item \textbf{正则太弱/太强}:调 \texttt{attention\_loss\_scale},观察分类准确率与 rob\_loss 的平衡。
\end{itemize}

\section{常见疑问速查}
\begin{itemize}[leftmargin=2em]
    \item \textbf{Q:为什么要先 Slot 再 Cross?}\\
    A:Slot 负责找“代表”。Cross 负责让每个样本在代表的帮助下更稳定。缺一不可。
    \item \textbf{Q:和 Gated Attention(gated-att)那套相比?}\\
    A:gated-att 是“单层加权平均”,计算轻;Slot+Cross 支持多子簇,结构更丰富,但也更复杂。
    \item \textbf{Q:如果 batch 很小呢?}\\
    A:类内样本少时,类级门会把权重压小,避免噪声统计影响训练。
\end{itemize}

\section{你可以做的下一步}
\begin{itemize}[leftmargin=2em]
    \item 对每个类画出责任分布($r_{mk}$)和 slot mass,看队长是否覆盖均匀。
    \item 记录训练过程中的 rob\_loss、intra\_mse 曲线,确认正则在起作用。
    \item 结合新的流程图(见文档同目录下的 PDF/PNG),在组会上逐阶段解释,帮助团队理解这套机制。
\end{itemize}

\end{CJK*}
\end{document}

\end{CJK*}
\end{document}
