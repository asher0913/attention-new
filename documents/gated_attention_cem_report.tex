\documentclass[12pt]{article}
\usepackage{CJKutf8}
\usepackage{amsmath, amssymb, amsfonts}
\usepackage{geometry}
\usepackage{hyperref}
\usepackage{enumitem}
\geometry{a4paper, margin=1in}
\hypersetup{
    colorlinks=true,
    linkcolor=blue,
    urlcolor=cyan
}
\setlength{\parskip}{0.6em}
\setlength{\parindent}{2em}

\title{Gated Attention 条件熵代理易懂手册}
\author{自动生成}
\date{\today}

\begin{document}
\begin{CJK*}{UTF8}{gbsn}
\maketitle

\begin{abstract}
这份文档面向第一次接触 \texttt{gated-att/} 的同学,目标是把 Gated Attention 条件熵代理讲成“看完就懂、上手就能用”的版本。我们会用通俗比喻串联整条链路:为什么要加这个模块、每一步在代码里的位置、它和 Slot+Cross 方案的区别、调参时应该先动哪个旋钮。你可以把它当作一篇讲稿/说明书,直接拿去给团队做分享。
\end{abstract}

\tableofcontents

\section{一张图先感知:我们在解决什么?}
\begin{itemize}[leftmargin=2em]
    \item \textbf{目的}:让同一类别的特征更集中(条件熵更低),减少模型泄露细节的风险。
    \item \textbf{做法}:不做复杂聚类,只学一个“注意力分布”,挑出类内真正关键的样本,按权重算出“类内波动大不大”。
    \item \textbf{位置}:所有逻辑都在 \texttt{gated-att/model\_training\_paral\_pruning.py} 的 \texttt{GatedAttentionPooling} 和 \texttt{GatedAttentionCEM} 中。
\end{itemize}

\section{整体故事:像是给班级点名}
\begin{enumerate}[leftmargin=2em]
    \item \textbf{先给每个班(类别)归好队}:把同一标签的特征拿出来。
    \item \textbf{点名 + 权重}:Gated Attention 会给每个学生一个“重要程度”。被判定为关键的样本权重高,普通样本权重低。
    \item \textbf{算班里的平均表现和波动}:根据权重算加权均值、加权方差,看看班里到底稳不稳。
    \item \textbf{超过阈值就扣分}:如果某个维度波动太大超过阈值,就对总损失贡献一个正则惩罚。
    \item \textbf{加权汇总到总损失}:班里人多的影响更大,人少的影响更小。
\end{enumerate}

\section{组件 1:GatedAttentionPooling(谁是关键样本?)}
\subsection*{3.1 直觉解释}
这一步像是给班里的每个学生打一个“关注度分数”。模型通过两个分支来决定:
\begin{itemize}[leftmargin=2em]
    \item $\tanh$ 分支($W_V$)告诉我们:“这个学生的特征方向怎么看?”
    \item $\sigma$ 分支($W_U$)像一扇门:“我要不要放大/缩小这名学生的影响?”
\end{itemize}
两个分支相乘后,再映射成一个标量,最后经过 softmax,得到一组权重 $a$,保证所有权重加起来是 1。

\subsection*{3.2 和代码对照}
\begin{itemize}[leftmargin=2em]
    \item 定义位置:第 17--44 行。
    \item 输入尺寸:[样本数 $M$, 特征维 $D$]。
    \item 输出:形状 [M, 1] 的 softmax 权重。
\end{itemize}

\section{组件 2:GatedAttentionCEM(用权重来判定是否“稳”)}
\subsection*{4.1 步骤拆解}
\begin{enumerate}[leftmargin=2em]
    \item \textbf{LayerNorm}:防止某些维度异常大造成偏差。
    \item \textbf{调用池化}:拿到上一节的权重 $a$。
    \item \textbf{算加权均值 $\mu_c$}:像做“班级平均成绩”,只是每个人成绩乘以他的权重。
    \item \textbf{算加权方差 $\sigma_c^2$}:看加权后的样本偏离均值多少。
    \item \textbf{防止数值爆炸}:用 \texttt{eps} 把方差下界住,再取 $\log$。
    \item \textbf{阈值判定}:如果 $\log\sigma_c^2$ 超过阈值,就记入惩罚;否则就是 0。阈值跟 \texttt{var\_threshold}、\texttt{reg\_strength} 有关。
    \item \textbf{平均到类级指标 $L_c$}:对所有维度取平均,得到这个班级的“惩罚分”。
    \item \textbf{计算类内 MSE}:只是为了日志监控,和梯度无关。
    \item \textbf{按样本占比加权}:班里人越多,在总损失里占比越大。
\end{enumerate}

\subsection*{4.2 关键变量一览}
\begin{itemize}[leftmargin=2em]
    \item \texttt{var\_threshold}:阈值基准,越大越宽容。
    \item \texttt{reg\_strength}:来自主程序的正则强度,和阈值一起决定严不严格。
    \item \texttt{attention\_loss\_scale}:训练时会乘在 rob\_loss 上,控制梯度力度。
\end{itemize}

\section{训练循环里怎么接入?}
\subsection*{5.1 触发条件}
\begin{itemize}[leftmargin=2em]
    \item \textbf{三个开关}:不是随机中心、$\lambda > 0$、当前 epoch 超过 warmup。
    \item \textbf{隐藏层宽度}:自动取 $\min(512, \max(64, D/4))$,不用手调。
    \item \textbf{参数注册}:第一次建模块时会把参数加到优化器里。
\end{itemize}

\subsection*{5.2 梯度流程}
\begin{enumerate}[leftmargin=2em]
    \item rob\_loss 先反向一次,把梯度存下来。
    \item 清梯度,对主任务反向。
    \item 把正则梯度按学习率缩放后加回编码器参数。
    \item 注意力模块的梯度直接累加。
\end{enumerate}

\subsection*{5.3 容错机制}
\begin{itemize}[leftmargin=2em]
    \item rob\_loss 或 intra\_mse 出现 NaN/Inf,会被置 0。
    \item 如果类内样本数 $\leq 1$,直接跳过,保证梯度路径连续。
\end{itemize}

\section{Slot+Cross vs. Gated Attention:该用谁?}
\subsection*{6.1 快速对比表}
\begin{itemize}[leftmargin=2em]
    \item \textbf{Slot+Cross}:\\
    优点——能建模多个子簇,对复杂数据更有力;\\
    缺点——结构复杂,算力和调参成本更高。
    \item \textbf{Gated Attention}:\\
    优点——结构简单、只有一个注意力池化,部署轻量;\\
    缺点——天然偏向“单簇”,对类内非常 multimodal 的数据可能不够精细。
\end{itemize}

\subsection*{6.2 建议}
\begin{itemize}[leftmargin=2em]
    \item 如果设备有限或想优先验证思路,可以先用 Gated Attention。
    \item 如果发现类内仍旧很散,考虑切换或加入 Slot+Cross。
\end{itemize}

\section{调参/排障套路}
\subsection*{7.1 常调参数}
\begin{itemize}[leftmargin=2em]
    \item \texttt{attention\_loss\_scale}:直接影响正则力度;训练稳定后可慢慢调大。
    \item \texttt{var\_threshold}:阈值大,惩罚少;阈值小,惩罚多。
    \item \texttt{eps}:如果 log 里还是出现 NaN,可以适当增大。
\end{itemize}

\subsection*{7.2 日志里要看什么?}
\begin{itemize}[leftmargin=2em]
    \item \textbf{rob\_loss 曲线}:应该随着训练逐渐稳定,不建议直接拉到很大。
    \item \textbf{intra\_mse}:如果一直不降,说明注意力权重没抓对关键样本。
    \item \textbf{注意力权重分布}:可以打印权重直方图,看看是否过于平均或过于集中。
\end{itemize}

\section{常见问题速查}
\begin{itemize}[leftmargin=2em]
    \item \textbf{Q:权重会不会全都平均?}\\
    A:初始可能接近平均,但随着训练,$W_V$、$W_U$ 会把关键样本权重大幅拉升。
    \item \textbf{Q:阈值怎么选?}\\
    A:默认值通常够用,如果你希望正则更激进,可以把 \texttt{var\_threshold} 调小一些。
    \item \textbf{Q:怎么知道正则有没有帮到隐私?}\\
    A:除了看 rob\_loss,还可以在攻击测试里比较是否更难重构。同时关注主任务准确率,避免过正则。
\end{itemize}

\section{下一步可以做什么?}
\begin{itemize}[leftmargin=2em]
    \item 结合同目录下的流程图(PDF/PNG),向同事展示每一步的输入输出。
    \item 记录每个 epoch 的 rob\_loss、intra\_mse、准确率,做一张曲线图观察趋势。
    \item 如果需要进一步解释,可以把注意力权重打印成热力图,让团队直观看到“谁最重要”。
\end{itemize}

\end{CJK*}
\end{document}

\end{CJK*}
\end{document}
