\documentclass[12pt]{article}
\usepackage{CJKutf8}
\usepackage{amsmath, amssymb, amsfonts}
\usepackage{geometry}
\usepackage{hyperref}
\usepackage{enumitem}
\geometry{a4paper, margin=1in}
\hypersetup{
    colorlinks=true,
    linkcolor=blue,
    urlcolor=cyan
}
\setlength{\parskip}{0.6em}
\setlength{\parindent}{2em}

\title{Gated Attention 条件熵代理框架解析报告(Original)}
\author{自动生成}
\date{\today}

\begin{document}
\begin{CJK*}{UTF8}{gbsn}
\maketitle

\begin{abstract}
本文面向 \texttt{gated-att/} 项目中的 \texttt{GatedAttentionCEM} 模块,系统梳理门控注意力替代 GMM/KMeans 的条件熵代理。内容包括目标动机、GatedAttentionPooling 的细节、加权统计与阈值化流程,以及在主训练循环中的集成方式。该版本保持原报告的数学表达,便于与流程图联动展示。
\end{abstract}

\tableofcontents

\section{背景与动机}
\subsection{条件熵最小化}
类似 Slot+Cross 框架,这里希望压缩类内特征分布,使条件熵 $H(Z \mid Y)$ 降低,以削弱攻击者在割点上的重建能力。传统 GMM/KMeans 代理在高维/小批量情况下不稳定,因此采用门控注意力直接学习类内加权统计。

\subsection{为什么用 Gated Attention}
\begin{itemize}[leftmargin=2em]
    \item 结构更轻:单层 gated attention 即可得到权重,不需要 slots 或多轮竞争,适合设备受限或想快速验证思路的场景。
    \item 仍可学到“关键样本”:注意力权重会偏向代表性强的样本,用加权方差来衡量类内散度。
    \item 直接替换 GMM:把 attention 权重理解为“软聚簇概率”,允许端到端训练。
\end{itemize}

\section{组件详解}
\subsection{GatedAttentionPooling(§17--44 行)}
\begin{enumerate}[leftmargin=2em]
    \item 输入为同类样本矩阵 $X_c \in \mathbb{R}^{M \times D}$。
    \item 通过两条支路:
    \[
        V = \tanh(W_V X_c), \qquad U = \sigma(W_U X_c),
    \]
    其中 $\tanh$ 捕捉方向,$\sigma$ 充当门控。
    \item 两条支路逐元素相乘后映射到标量 logits:
    \[
        \ell = W_w (V \odot U) \in \mathbb{R}^{M \times 1}.
    \]
    \item 沿样本维度做 softmax 得到注意力权重 $a = \mathrm{softmax}(\ell)$,满足 $\sum_m a_m = 1$。
\end{enumerate}
权重 $a$ 中较大的样本被认为是类内关键信息,后续加权均值/方差都以此为系数。

\subsection{GatedAttentionCEM(§46--118 行)}
对每个类 $c$:
\begin{enumerate}[leftmargin=2em]
    \item LayerNorm:$\hat{X}_c = \mathrm{LayerNorm}(X_c)$,对齐尺度。
    \item 通过 pooling 得到注意力 $a$。
    \item 加权均值:
    \[
        \mu_c = \sum_m a_m \hat{x}_m.
    \]
    \item 加权方差:
    \[
        \sigma_c^2 = \sum_m a_m (\hat{x}_m - \mu_c)^2,
    \]
    并以 $\mathrm{eps}$ 下界避免 $\log 0$。
    \item 阈值化 log-variance:
    \[
        \log\sigma_c^2 = \log(\sigma_c^2 + \gamma), \qquad
        \mathrm{ce\_sur} = \max\bigl(0,\; \log\sigma_c^2 - \log\sigma^2_{\text{thr}}\bigr),
    \]
    其中 $\sigma^2_{\text{thr}} = \max(\texttt{var\_thr} \cdot \texttt{reg\_strength}^2, 10^{-8}) + \gamma$。
    \item 类级输出 $L_c = \mathrm{mean}(\mathrm{ce\_sur})$,并记录 MSE 作为监控指标。
    \item 按样本占比 $w_c = M / B$ 加权,加到总的 $rob\_loss$ 与 $intra\_mse$ 中。
\end{enumerate}

\section{训练循环中的集成}
\begin{itemize}[leftmargin=2em]
    \item 在 `train_target_step` 中,若满足 λ>0、epoch>warmup、非随机中心,就会调用 `GatedAttentionCEM` 计算 `rob_loss`(`model_training_paral_pruning.py:742-902`)。
    \item 输出的 `rob_loss` 乘以 `attention_loss_scale` 并先反向保存梯度,再与主任务(交叉熵等)梯度合成。
    \item NaN/Inf 的结果会清零,保证主干训练不被破坏。
\end{itemize}

\section{数值稳定性与调参}
\begin{itemize}[leftmargin=2em]
    \item \texttt{hidden\_dim}:注意力 MLP 的隐藏维度,自动取 $[64, 512]$。
    \item \texttt{var\_threshold}, \texttt{reg\_strength}:决定 log-variance 阈值;阈值越大越宽松。
    \item \texttt{attention\_loss\_scale}:控制正则梯度大小;训练稳定后可调大。
    \item LayerNorm + $\mathrm{eps}$ 防止数值问题,类内样本不足时则跳过该类。
\end{itemize}

\section{结论}
Gated Attention CEM 用单层 gated pooling + 阈值化 log-variance 的方式,替代 GMM/KMeans 作为类内条件熵 surrogate。结构轻量、易部署,适合在算力有限或类内单模态场景下优先使用。该报告与流程图(`gated_attention_cem_flowchart_original.v6.pdf`)配套,可直接用于展示。

\end{CJK*}
\end{document}
