\documentclass[12pt]{article}
\usepackage{CJKutf8}
\usepackage{amsmath, amssymb, amsfonts}
\usepackage{geometry}
\usepackage{hyperref}
\usepackage{enumitem}
\geometry{a4paper, margin=1in}
\hypersetup{
    colorlinks=true,
    linkcolor=blue,
    urlcolor=cyan
}
\setlength{\parskip}{0.6em}
\setlength{\parindent}{2em}

\title{Slot + Gated Cross CEM 易懂手册(v6)}
\author{自动生成}
\date{\today}

\begin{document}
\begin{CJK*}{UTF8}{gbsn}
\maketitle

\begin{abstract}
这份 v6 版手册对应根目录的 \texttt{SlotCrossAttentionCEM},配套流程图 \texttt{slot\_gated\_cem\_flowchart\_original.v6.pdf}。我们保持原有流程(Slot Attention → Gated Cross Attention → 混合槽统计),但把解释方式改成“第一手就能看懂”的结构,方便直接拿去做 PPT 或讲稿。
\end{abstract}

\tableofcontents

\section{一句话背景}
\begin{itemize}[leftmargin=2em]
    \item \textbf{目标}:让同一类别的特征在中间层更紧凑,降低条件熵,减小被攻击者还原的风险。
    \item \textbf{方法}:用“队长 + 向队长学习”来概括类内多样性,再用多重门控判定哪些方差值得惩罚。
    \item \textbf{输出}:一个正则损失 \texttt{rob\_loss}(用于反向)和一个监控指标 \texttt{intra\_mse}。
\end{itemize}

\section{总览:四个阶段,像排班表}
\begin{enumerate}[leftmargin=2em]
    \item \textbf{阶段 1:Slot Attention 挑队长}\\
    同类样本站成一排,多个 slots(队长)和大家多轮互动后,成为几个子簇代表。
    \item \textbf{阶段 2:Gated Cross Attention 向队长取经}\\
    每个样本根据注意力权重向队长借经验,门控残差确保逐步引入新信息。
    \item \textbf{阶段 3:混合槽统计判断“是否稳”}\\
    用余弦相似度分配责任,计算加权均值/方差,再通过维度门、SNR 门、Softplus 阈值和槽权重过滤噪声。
    \item \textbf{阶段 4:类级聚合并输出损失}\\
    对每类得到的结果乘以样本占比,合成为 \texttt{rob\_loss};同时记录 \texttt{MSE} 曲线。
\end{enumerate}

\section{阶段细节}
\subsection{1. Slot Attention(队长竞选)}
\begin{itemize}[leftmargin=2em]
    \item \textbf{输入形状}:$[B=1, N=M, D]$,一次只处理某个类别的样本。
    \item \textbf{流程}:
    \begin{enumerate}[leftmargin=1.8em]
        \item LayerNorm 后映射成 Key/Value。
        \item slots 从可学习的高斯分布采样,避免完全一样。
        \item 多轮循环:
        \begin{itemize}
            \item 用 slots 发出 Query,和样本 Key 点积 → softmax 责任。
            \item 按责任加权 Value,得到更新量。
            \item 用 GRU 和 MLP 更新 slots(保留记忆 + 引入新信息)。
        \end{itemize}
    \end{enumerate}
    \item \textbf{额外细节}:点积除以 $(d)^{1/4}$ 防止梯度爆炸,LayerNorm 让不同 batch 可比。
\end{itemize}

\subsection{2. Gated Cross Attention(向队长取经)}
\begin{itemize}[leftmargin=2em]
    \item \textbf{输入}:原样本特征 $x_m$ 和上一步的 slots。
    \item \textbf{流程}:
    \begin{enumerate}[leftmargin=1.8em]
        \item 样本做 Query,slots 做 Key/Value。
        \item 计算注意力后得到增强特征。
        \item 通过 $\tanh(\alpha_{\text{xattn}})$ 门控残差加入原特征。
        \item 再通过 Pre-LN + FFN + $\tanh(\alpha_{\text{ffn}})$ 的门控残差。
    \end{enumerate}
    \item \textbf{意义}:门控让模型一开始保持原状,训练稳定后再逐步依赖注意力。
\end{itemize}

\subsection{3. 混合槽统计(判断是否松散)}
\begin{enumerate}[leftmargin=2em]
    \item \textbf{责任分配}:增强后的样本与 slots 做余弦相似度,softmax 得到 $r_{mk}$。
    \item \textbf{加权统计}:用 $r_{mk}$ 计算每个槽的加权均值 $\mu_s$、方差 $\sigma_s^2$。
    \item \textbf{三种门控}:
    \begin{itemize}
        \item \textbf{维度软门}:LayerNorm + MLP + Sigmoid,抑制噪声维度。
        \item \textbf{SNR 硬门}:$\sigma^2 / (\mu^2+\varepsilon)$,信噪比低的压低。
        \item \textbf{Softplus 阈值}:对 $\log\sigma^2$ 做平滑阈值,避免 ReLU 的断点。
    \end{itemize}
    \item \textbf{槽权重}:根据 slot mass 调整权重,代表性强的槽更重要。
    \item \textbf{类级聚合}:门控项相乘 → 按槽权重求和 → 对维度取平均,得到 $L_c$;再乘以类级 Sigmoid 门(batch 样本少则权重低)。
\end{enumerate}

\section{训练中的使用姿势}
\begin{itemize}[leftmargin=2em]
    \item \textbf{触发条件}:$\lambda>0$、非随机中心、epoch 超过 \texttt{attention\_warmup\_epochs}。
    \item \textbf{梯度流程}:
    \begin{enumerate}[leftmargin=1.8em]
        \item rob\_loss 先反向,保存编码器/注意力梯度。
        \item 清梯度,再对交叉熵反向。
        \item 按学习率缩放 \& $\lambda$ 加回 rob\_loss 的梯度。
    \end{enumerate}
    \item \textbf{保护机制}:早期关断、NaN/Inf 自动置零,确保不会把主任务拉崩。
\end{itemize}

\section{可调旋钮与排障}
\subsection*{6.1 常用参数}
\begin{itemize}[leftmargin=2em]
    \item \texttt{num\_slots}:默认 8,可按类内复杂度调。
    \item \texttt{num\_iterations}:默认 3,越大越精细但越慢。
    \item \texttt{attention\_loss\_scale}:默认 0.25,控制正则强度。
    \item \texttt{var\_threshold}:越小越严格,越大越宽松。
\end{itemize}

\subsection*{6.2 常见问题}
\begin{itemize}[leftmargin=2em]
    \item \textbf{rob\_loss 一直是 0?}\\
    可能还在 warmup,或门值触发了早期关断;看日志里的 \texttt{[CEM-GATE]} 提示。
    \item \textbf{rob\_loss 过大导致精度掉}?\\
    先减小 \texttt{attention\_loss\_scale};也可以调大 \texttt{var\_threshold}。
    \item \textbf{NaN/Inf}?\\
    检查输入特征是否已有 NaN;或调大 Softplus 边际、eps。
\end{itemize}

\section{拿着这份文档做什么?}
\begin{itemize}[leftmargin=2em]
    \item 将本手册和 \texttt{slot\_gated\_cem\_flowchart\_original.v6.pdf} 一起做成 PPT。
    \item 在组会上按“阶段概述 → 细节 → 调参”三段讲法分享。
    \item 如需英文版或更技术化版本,可在此基础上补充数学公式。
\end{itemize}

\end{CJK*}
\end{document}
